\chapter{Analyse von artverwandten Spielen zur Konzeptentwicklung}\label{sec:analysis}

Das grundlegende Spielkonzept von Connecting-Minds wird durch die Analyse verwandter Spiele, wie sie im Kapitel \emph{\nameref{sec:sota}} beschrieben sind, sowie durch weitere Spiele mit vergleichbarem Spieldesign ergänzt. Ziel ist es, durch diese vergleichende Betrachtung Gestaltungsprinzipien und Mechaniken zu identifizieren, die als Inspiration und konzeptionelle Grundlage für die Entwicklung von Connecting-Minds dienen können.

Darüber hinaus sollen die Ergebnisse dieser Analyse zur Beantwortung der zentralen Forschungsfrage, \emph{\say{Welche spezifischen Eigenschaften muss eine solche Umgebung aufweisen und welche Kommunikationsparameter werden dabei angesprochen?}}, beitragen.

\section{Methodik}
Zur systematischen Analyse der artverwandten Spiele wurde ein mehrstufiges, eigens entwickeltes Analyseraster herangezogen. Dieses Raster gliedert sich in vier aufeinander aufbauende Abschnitte:

\subsection{Visuelle Analyse}
Im ersten Schritt wurde der Fokus auf die sichtbaren Rätsel- und Hinweiselemente innerhalb des visuellen Designs der untersuchten Spiele gelegt. Relevante Elemente wurde direkt im Bildmaterial markiert, um sie im Rahmen der nachfolgenden Auswertung gezielt untersuchen zu können. Diese Markierungen dienten der Identifikation wiederkehrender Gestaltungselemente sowie der Kategorisierung der eingesetzt Rätseldesigns. Darüber hinaus konnten auf diese Weise Interaktionsflüsse sichtbar gemacht und chronologisch geordnet werden. Zusätzlich wurden die jeweiligen Kamera- und Perspektivführungen sowie Ansichten auf die Spielumgebung berücksichtigt, um Rückschlüsse auf das räumliche Design ziehen zu können.

\subsection{Erstellung eines Diagramms zur Rätselstruktur}
Anschließend wurden für ausgewählte Spielabschnitte oder komplette Spielverläufe UML-Ablaufdiagramme erstellt. Ziel war es, die strukturelle Organisation sowie die logische Verknüpfung und Verschachtelung der Rätselinhalte zu erfassen. Die Vorgehensweise orientiert sich an methodischen Ansätzen zur Rätselstrukturierung, wie sie bspw. von \cite{schafer_grim_1996} vorgeschlagen wurden.

\subsection{Deskriptive Übertragung}
Im dritten Schritt erfolgt eine deskriptive Zusammenführung der aus der visuellen Analyse und der strukturellen Diagrammerstellung gewonnenen Erkenntnisse. Diese wurden hinsichtlich ihrer Implikationen für das Design und die Gestaltung von Rätseln systematisch beschrieben.

\subsection{Schlussfolgerung}
Abschließend wurden die zuvor gewonnen Beobachtungen in Beziehung zur Konzeptentwicklung von Connecting-Minds gesetzt. Dabei wurde untersucht, welche Elemente potenziell übernommen, angepasst oder vermieden werden sollten. Diese Rückschlüsse flossen gezielt in die Weiterentwicklung der eigenen Konzepts ein.

\section{Auswahl der Stichprobe}
Die Auswahl der Spiele, die im Rahmen der Analyse näher untersucht wurden, erfolgt auf Grundlage spezifischer Kriterien und dient der gezielten Konzeptentwicklung für Connecting-Minds. Dabei wurden nicht sämtliche in Kapitel \ref{sec:sota} vorgestellten Spiele berücksichtigt. Ergänzend wurden weitere relevante Spiele in die Analyse einbezogen, die im vorherigen Kapitel nicht behandelt wurden, aber im Ergebnisabschnitt eingeführt und kontextualisiert werden.

Der Fokus der Analyse liegt auf Spielen, die über unterschiedliche Geräte oder Anwendungen hinweg gespielt werden müssen oder durch ihre asymmetrischen Rollen asymmetrische Interaktionen erzwingen. Aufgrund der konzeptionellen Ausrichtung von Connecting-Minds als asymmetrisches Multiplayer-Spiel wurde eine Auswahl getroffen, die besonders geeignete Beispiele für unterschiedliche Rollenverteilungen und Kommunikationsanforderungen im kooperativen Spielkontext liefert.

Konkret wurden die Spielreihe We were here sowie das Spiel Myrmidon als zentrale Analysekandidaten ausgewählt. Beide Titel bieten durch ihre Spielmechanik und das Rollenverständnis eine fundierte Grundlage für die Untersuchung asymmetrischer und kooperativer Multiplayerspiele. Ergänzt wird diese Auswahl durch das Indie-Adventure-Spiel \say{Tiny Room Stories: Town Mystery}, das, obwohl es ein Singleplayerspiel ist, durch seine strukturierte Rätselgestaltung Vorlagen für die Konzeption von Connecting-Minds liefert.

Von der Analyse ausgeschlossen wurden hingegen Spiele wie die Titel des Entwicklerstudios Hazelight, da deren Splitscreen-Mechanik es den Spielenden erlaubt, die Perspektive des Gegenübers unmittelbar nachzuvollziehen. Dadurch entfällt der Teil der Kommunikation zur Beschreibung des Spielgeschehens. Durch dieses zentrale Element in Connecting-Minds soll die Kommunikation provoziert werden. Ebenso wurde das Spiel Keep Talking and Nobody Explodes nicht in die engere Analyse aufgenommen, da hier lediglich eine Person aktiv im Spiel agiert, während die übrigen Teilnehmer außerhalb der Anwendung interagieren. Dieses Spielmodell wurde als nicht hinreichend relevant für die geplanten Designziele eingestuft und dient daher als Ausschlusskriterium. Das Spiel The Past Within wurde ebenso aus dem Analysekreis ausgeschlossen, da dessen struktureller Aufbau nicht mit der angestrebten Weiterentwicklung von Connecting-Minds vereinbar war. Im Spiel lösen die Spieler zunächst unabhängig voneinander Rätsel innerhalb ihrer eigenen Anwendung. Erst an bestimmten Punkten greift das Spielgeschehen ineinander, sodass eine kooperative Abstimmung notwendig wird. Diese Gestaltung führt jedoch dazu, dass der Bedarf an kontinuierlicher Kommunikation nur punktuell entsteht, da zunächst jeweils überprüft werden muss, ob die Unterstützung des Mitspielers überhaupt erforderlich ist.

\section{Ergebnisse der Analysen}

Im nächsten Schritt werden die ausgewählten Spiele We were here, We were here too, Tiny Room Stories und Myrmidon hinsichtlich ihrer relevanten Merkmale analysiert. Die Analyse erfolgt nach dem zuvor vorgestellten Schema und beleuchtet insbesondere die Darstellungen im Spiel, sowie die Rätselmechaniken und die daraus resultierenden kommunikativen Aspekte.

\subsection{We were here - Spielreihe}
In der vorliegenden Analyse wurde der Fokus auf die ersten beiden Teile der Spielreihe, We were here und We were here too, beschränkt. Eine umfassende Untersuchung aller Titel der Reihe hätte den Rahmen dieser Arbeit überschritten, während der zusätzliche Erkenntnisgewinn im Verhältnis zum Aufwand als gering einzuschätzen ist. Die ausgewählten Spiele wurden daher in erste Linie in der Breite untersucht und nicht der Menge, um grundlegende Strukturen der Interaktion und Kommunikation zu erfassen. Im Folgenden werden die beiden Titel gemeinsam betrachtet und als eine zusammengehörige Einheit analysiert.

\paragraph{Visuelle Analyse}
Die Abbildungen im Anhang \ref{sec:append_anylsis_wwh_wwht_visual}: \nameref{sec:append_anylsis_wwh_wwht_visual} geben einen Einblick in die Analyse der Spielerrollen in den beiden Spielen. Auffällig ist, dass die Rätselelemente überwiegend durch visuelle Elemente umgesetzt sind. Es müssen Symbole und bildhafte Darstellungen korrekt identifiziert und zugeordnet oder in einer bestimmten Weise platziert werden. Die Rätsel sind dabei integraler Bestandteil der Spielwelt. Hinweise erscheinen entweder als Symbole an Wänden oder als in die Umgebung eingebettete Textelemente. Nur in Ausnahmefällen erfolgt die Vermittlung der zu Lösung nötigen Informationen über auditive Elemente oder Texte in Büchern.

In We were here zeigt sich eine klare Rollenverteilung zwischen dem \say{Explorer}, der innerhalb der Spielwelt agiert und direkt mit den Rätseln konfrontiert ist, und dem \say{Librarian}, der sich in einem zentralen Hub aufhält, in dem er über die notwendigen Informationen zu den Lösungen verfügt. Der Spielfortschritt ist demnach auf eine enge zielgerichtete Kommunikation zwischen den beiden Rollen angewiesen. Ein ähnliches Muster zeigt sich in  We were here too. Der \say{Peasant} befindet sich dabei zunächst in rätselhaften Gewölben, während der \say{Lord} die korrespondierenden Informationen erhält und ebenfalls in rätselhaften Gewölben feststeckt. Im Vergleich zum ersten Teil wird letzterer jedoch häufiger aktiv in das Lösen von Rätseln eingebunden und bewegt sich ebenfalls durch verschiedene Räume, anstatt dauerhaft in einem zentralen Hub zu verbleiben.

\paragraph{Analyse des Rätseldesigns}
Die Abbildung im Anhang \ref{sec:append_riddles_wwh_wwht}: \nameref{sec:append_riddles_wwh_wwht} zeigen die Ablaufdiagramme der beiden Spiele We were here und We were here too. Auffällig ist, dass im ersten Teil die Rolle des Librarian im Kontext des Rätseldesigns einem weitgehend linearen Verlauf folgt. Verschachtelte oder mehrstufige Rätselstrukturen treten nur selten auf. Die Spielstruktur sieht in der Regel vor, dass der Explorer mit den eigentlichen Rätseln und Hindernissen in der Spielwelt konfrontiert wird, während der Librarian außerhalb dieser Räume agiert und die benötigten Informationen zur Lösung bereitstellt. Die Interaktion zwischen den Rollen ist dadurch überwiegend eindimensional und von einer klaren Aufgabenverteilung geprägt.

Im Nachfolger We were here too lassen sich im Vergleich vermehrt verschachtelte oder verkettete Rätselstrukturen erkennen, etwa in Raum zwei des Spiels. Solche komplexeren Designs bleiben jedoch die Ausnahme. Insgesamt wiederholt sich ein ähnlicher Aufbau wie im Vorgängerspiel. Die Rollenverteilung bleibt asymmetrisch, die Zusammenarbeit erfolgt primär über Informationsaustausch, und die Kommunikationsstruktur folgt weiterhin einem einfachen Muster aus Anfrage und Antwort.

\paragraph{Schlussfolgerung}
Der zentrale Aspekt der Kommunikationsaufforderung ist in beiden Spielen erkennbar. Er bildet eine gemeinsame Grundlage. In We were here fällt jedoch auf, dass das Rätseldesign häufig monoton und eindimensional wirkt. Der Explorer stößt auf Rätsel, beschreibt gegebene und/oder gesuchte Gegenstände, der Librarian geht in einen bestimmten Raum und beschreibt gesuchte Gegenstände, der Explorer löst Rätsel. Es existieren zwar einzelne Ausnahmen, in denen der Librarian aktiv werden muss um dem Explorer weiterzuhelfen, etwa als der Librarian das richtige Ventil vom Rohr öffnen musste. Doch diese wechselseitige Abhängigkeit bleibt die Ausnahme. Gerade diese Form der Verzweigung ist für Connecting-Minds essenziell und sollte eine tragende Rolle Spielen. Auf dieser Grundidee lässt sich aufbauen, auch wenn nicht jedes Rätsel in seiner konkreten Ausgestaltung überzeugt, bieten bestimmte Ansätze dennoch wertvolle Anregungen, besonders im Hinblick auf die Möglichkeit, abwechslungsreiche, variantenreichere und stärker auf Kooperation ausgelegte Rätsel für Connecting-Minds zu entwickeln.

Deutlich weiter geht das zweite Spiel, We were here too, in dem die gegenseitige Verflechtung zwischen den beiden Rollen wesentlich häufiger zum Tragen kommt. In vielen Fällen ist es der Peasant, der nicht nur seinen eigenen Weg zum nächsten Raum freischaltet, sondern zugleich auch den vom Lord. In zwei Fällen ist das Prinzip sogar umgekehrt gestaltet. Der Lord ermöglicht dem Peasant den Zugang zu neuen Bereichen. In einem der beiden Fällen muss der Peasant eine Wendeltreppe hinaufrennen, unter der sich der Boden langsam einzieht. Er kommt jedoch nur bis zu einer Speerwand, die über einen Würfel geöffnet werden kann. Er muss dem Lord den Aufschnitt des Würfels beschreiben, welcher den richtigen Würfel auswählen und in die Zielablage ablegen muss. Durch die Verschachtlung wird der Grundgedanke der Kommunikation und Zusammenarbeit deutlich verstärkt hervorgehoben und erzeugt ausgeglichenen Spaß in den Anwendungen. Ein weiterer Zusatz des Rätseldesigns ist das aufeinander Aufbauen von Rätseln. Dieses Element wird im Rahmen dieser Schlussfolgerung als \say{Mehrstufigkeit} bezeichnet und ist zum Beispiel in Raum zwei zu finden, bei dem aneinander gekettet, verschiedene Rätsel gelöst werden müssen.

Diese strukturelle Ausrichtung eignet sich grundsätzlich gut als gestalterische Vorlage, sofern sie sich sinnvoll in Connecting-Minds übertragen lässt. Gleichzeitig zielt Connecting-Minds auf eine noch tiefere und kontinuierliche Abhängigkeit zwischen den beiden Spielerrollen. Beide Rollen sollen nicht nur gelegentlich, sondern regelmäßig und systematisch aufeinander angewiesen sein. Die Zusammenarbeit wird damit zur unverzichtbaren Grundlage des Spielfortschritts. Aus der engen Verflechtung ergibt sich, dass Kooperation nicht nur hilfreich, sondern spielentscheidend wird.

Konzepte wie mehrstufige Rätsel oder Einsatz von zeitlichen Begrenzungen (Timer) sind in diesem Zusammenhang ebenfalls interessante Überlegungen. Während Mehrstufigkeit sich als zentrales Element für die Rätselgestaltung anbietet, sollte der Timer gezielt und situationsabhängig eingesetzt werden, da sein Einfluss stark vom jeweiligen Spannungsbogen und dem beabsichtigten Spielerlebnis abhängt.

\subsection{Tiny Room Stories: Town Mystery}
Tiny Room Stories: Town Mystery ist ein Escape-Room-Spiel mit eingebetteter Detektivgeschichte, für mobile Endgeräte (IOS und Android) sowie für den PC verfügbar. Die Spieler übernehmen die Rolle eines Privatdetektivs, der einem Hilferuf seine Vaters folgt und in die scheinbar verlassene Stadt  seines Vaters reist. Ziel ist es, das Verschwinden der Einwohner aufzuklären, indem verschiedene Orte in der Stadt untersucht, Hinweise gesammelt, Rätsel gelöst und mechanische wie logische Sperren überwunden werden. 
Das Spiel verbindet Elemente vom Escape-Room-Design mit Merkmalen von Point-and-Click-Adventures. Eine Besonderheit stellt die vollständig drehbare \ac{3D}-Umgebung dar, die es den Spielern erlaubt, die einzelnen Level aus unterschiedlichen Perspektiven zu betrachten und so versteckte Hinweise oder interaktive Objekte gezielt aufzuspüren (vgl. \citealp{kiary_games_tiny_2021}).

Im Gegensatz zur We were here-Reihe handelt es sich bei Tiny Room Stories: Town Mystery um ein reines Singleplayerspiel. Der Fortschritt des Spiels erfolgt durch das Freischalten von Abschnitten und narrativen Informationen, wodurch die Handlung schrittweise aufgedeckt wird. Für die Analyse wurden der Prolog sowie Kapitel 1 des Spiels betrachtet. Der grundlegende Aufbau bleicht auch in den folgenden Kapiteln erhalten. Veränderungen ergeben sich primär in der inhaltlichen Gestaltung sowie in der Komplexität und Verschachtelung der jeweiligen Rätsel und Hinweise.

\paragraph{Visuelle Analyse}
Das in Anhang \ref{sec:append_analysis_trstm}: \nameref{sec:append_analysis_trstm} dargestellte Ergebnis gibt einen Überblick über die erste Analyse des Spiels. Der Schwerpunkt liegt dabei zunächst auf den Informationen und Anweisungen die das Spiel dem Spieler zu Verfügung stellt. Darauf aufbauend erfolgt eine Kategorisierung der Rätseltypen sowie eine Betrachtung der jeweiligen Lösungsmechanismen. Ergänzend wurde auf die Art und Darstellung von Steuerungshinweisungen geachtet, um potenzielle Übertragungsmöglichkeiten für die Konzeptentwicklung von Connecting-Minds zu identifizieren. Das Spiel zeichnet sich insgesamt durch die Spielumgebung eingebetteter Rätsel und kontextuelle Hinweise aus. Diese werden durch Notizzettel und andere schriftliche Verweise ergänzt, die das narrative Rätseldesign unterstreichen. Beide Designansätze greifen ineinander und ermöglichen ein abwechslungsreiches sowie mitunter herausforderndes Spielerlebnis.

\paragraph{Analyse des Rätseldesigns}
Die Abbildung im Anhang \ref{sec:append_riddles_trstm}: \nameref{sec:append_riddles_trstm} zeigt das Ablaufdiagramm der Rätselstruktur des Spiels. Sowohl in der visuellen Gestaltung des Spiels als auch im Aufbau der Rätsel wird deutlich, dass der Zugang zu neuen Räumen regelmäßig neue Hinweise oder Rätsel freischaltet, die wiederum für den weiteren Fortschritt erforderlich sind. Die einzelnen Rätsel sind dabei ineinander verschachtelt und interdependent. Das Lösen von Rätsel A erschließt oftmals Hinweise, die erst das Lösen von Rätsel B ermöglichen, das wiederum Voraussetzung für Rätsel C sein kann. Eine solche strukturierte Kettenlogik zeigt sich exemplarisch im Bücherregal-Rätsel in Kapitel eins sowie bereits im Prolog beim Öffnen der Schranke.

\paragraph{Schlussfolgerung}
Die Spielsteuerung des Spiels kann in mehrfacher Hinsicht eine sinnvolle Vorlage für die Konzeption von Connecting-Minds dienen. Einerseits überzeugt sie durch eine klare, intuitive Bedienbarkeit, die sich potenziell gut auf den geplanten Prototyp übertragen lässt. Andererseits weist die Struktur der Rätsel eine Logik auf, die sich für die Gestaltung asymmetrischer Spielerrollen besonders gut eignet.

Von besonderem Interesse ist hierbei die Verknüpfung zwischen untergeordneten und übergeordneten Rätseln. Erkenntnisse oder gelöste Elemente auf der einen Seite eröffnen erst die Möglichkeit, auf der anderen Seite weitere Fortschritte zu erzielen. Dieses Prinzip wechselseitiger Abhängigkeiten bietet ein hohes Potenzial für ein kooperatives Rätseldesign, bei dem beide Spieler aufeinander angewiesen sind. Es lassen sich vielschichtige Herausforderungen gestalten, die zentral für die angestrebte Wirkung in Connecting-Minds sind.

\subsection{Myrmidon}
Das Spiel Myrmidon reiht sich in die Spiele der asymmetrischen Multiplayer wie die We Were Here-Reihe ein. Es müssen kooperativ Wege freigelegt und ermöglicht werden. Die Analyse des Spiels umfasst das anfängliche Tutorial. Es unterscheidet sich lediglich in der szenischen Einbettung der Spielerrolle der animierten Puppe. Im Tutorial ist die Ausgestaltung der Welt identisch zu der, die der Animator sieht. Im Hauptteil des Spiel, das nur aus dem Tutorial und einem ersten richtigen Level besteht, wird die Szenerie der Puppe in einen Stop-Motion-Film eingebettet. Der Animator hingegen sieht eine interaktive Kulisse, mit wenig Dekoration.

\paragraph{Visuelle Analyse}
Die Abbildungen in Anhang \ref{sec:append_anylsis_m_visual}: \nameref{sec:append_anylsis_m_visual} zeigen die Ergebnisse der ersten Analyse. Nach dem Zusammenfinden in einer Lobby wählen die Spieler ihre jeweilige Rolle innerhalb einer kooperativen Dyade aus. Die Rollenverteilung ist dabei asymmetrisch. Eine Person übernimmt die Rolle des \say{Animators}, die andere die der \say{Stop-Motion Puppe}. Die visuelle Gestaltung orientieren sich dabei stark an einem stilisierten Stop-Motion-Film. 
Der \say{Animator} steuert dabei die Kamera in einem offenen, dreidimensionalen Raum und ist für die aktive Gestaltung der Kulisse verantwortlich. Diese Kulisse bildet den Bewegungsraum für die \say{Puppe}, deren Fortschritt vom Eingreifen des Animators abhängt. So muss der Animator im Tutorial durch das Öffnen von Schubladen eine provisorische Treppe errichten, damit die Puppe die höhergelegene Kisten und Schrankfächer erreichen kann. 
Manche der benötigten Elemente befinden sich zunächst nicht im Zugriff des Animators, sondern sind in verschiebbaren Kisten in der Spielwelt enthalten. Erst wenn diese Kisten durch die Puppe aus der Spielwelt hinausgeschoben wurden,  werden die enthaltenen Kulissenteile im Spielraum des Animators interagierbar. Dieser kann sie anschließend an vorbestimmten Bereichen, z. B. einer Korkwand, Haltestangen oder Falltüren platzieren, um der Puppe neue Wege zu ermöglichen.
Die Puppe hingegen wird in der Spielwelt eines Stop-Motion-Films bewegt. Sie sieht den Außenbereich, den der Animator sieht, nicht. Aufgrund limitierter Kameraansichten ist die Puppe an manchen Stellen auf die verbale Navigation des Animators angewiesen. Die Kommunikation zwischen beiden Rollen ist somit essenziell, da viele Hindernisse nur durch gemeinsames Planen und Koordinieren überwindbar sind.

\paragraph{Analyse des Rätseldesigns}
Das im Anhang \ref{sec:append_riddles_m}: \nameref{sec:append_riddles_m} abgebildete Diagramm zeigt das Rätseldesign des Tutorials. Auffallend ist, dass beide Spieler dem Weg der Puppe folgen müssen um an das Ziel des Abschnitts zu gelangen. Auf diesem Weg gerät die Puppe an Hindernisse, die der Animator durch das Interagieren mit der Spielkulisse beseitigen kann. Nachdem beide Spieler die Spielwelt betreten haben, muss der Animator seiner Puppe einzelne Schubladen in einem kleinen Schreibtischschränkchen öffnen, damit die Puppe die Schubladen als eine Art Leiter nutzen kann. Weitere Arten von \say{Rätsel} bzw. Hindernisse, die beseitigt werden müssen, sind mehrstufig verbunden. Das bedeutet, der Animator muss zwar der Puppe eine Bewegungsmöglichkeit in die Kulisse bauen, allerdings muss die Puppe dafür zunächst diese Bewegungsmöglichkeit freischalten. Dies geschieht in der Mitte des Tutorials, bei dem die Puppe zunächst eine Kiste aus der Spielwelt schieben muss, damit die enthaltene Stange an die Wand gesteckt und die Puppe darüber einen Abgrund überwinden kann.
Das letzte Hindernis im Tutorial besteht aus der Kombination aus Schubladen in der Kulisse und weiteren Hilfsmöglichkeiten, die die Puppe freischalten muss. Im weiteren Verlauf des Spiels wiederholen sich diese drei Arten von Rätsel. 

\paragraph{Schlussfolgerung}
Aus der Visuellen Analyse und der Analyse des Rätseldesigns geht hervor, dass einige Parallelen zur Spielidee von Connecting-Minds existieren. Die größte Parallele existiert im Freischalten und Nutzen von Gegenständen um gemeinsam Hindernisse zu überwinden. Die größten Unterschiede bestehen allerdings in der Darstellung des anderen Spieleravatars in der eigenen Anwendung. Im Connecting-Minds soll der Watcher den Avatar des Players in der Spielwelt nicht sehen. Es soll die Kommunikation durch das Beschreiben der derzeitigen Position gefördert werden. Außerdem ist die Hauptaufgabe des Animators auf das Unterstützen der Puppe beschränkt. Es existieren keine Abschnitte, in denen der Animator für sich Umgebungsrätsel oder Hindernisse lösen muss. Dieser Aspekt soll in Connecting-Minds für den Watcher umgesetzt werden. An diesen Abschnitten muss dann sogar der Player dem Watcher unterstützen. 

\section{Zusammenfassung und Interpretation der Ergebnisse}
Die Analyse der vier Vergleichsspiele zeigt, dass das Zusammenspiel zwischen asymmetrischen Rollen, sowie die Gestaltung kommunikationsfördernder Rätseln  bereits in einigen Spielen enthalten sind, allerdings auch gute Vorlagen bieten können um neue Ideen zu entwickeln.
Die größte Relevanz für das Konzept von Connecting-Minds ergibt sich aus der engen wechselseitigen Abhängigkeiten der Spielerrollen, wie sie insbesondere in We Were Here Too sowie Myrmidon erkennbar sind.

Während We Were Here nur punktuell auf gegenseitige Interaktion setzt und häufig ein eher eindimensionales Rätselmuster verfolgt, gelingt es dem Nachfolger We Were Here Too, die strukturellen Defizite aufzugreifen und diese durch eine stärkere verzahnte Aufgabenverteilung zu ergänzen. Die wechselseitigen Bedingtheit von Fortschritt erzeugt nicht nur eine höhere kommunikative Dichte, sondern verankert Kooperation als auch spielentscheidende Elemente. Konzepte wie Mehrstufigkeit und temporale Elemente (z. B. Timer) erscheinen in diesem Kontext als potenzielle Designprinzipien für Connecting-Minds, deren gezielter Einsatz sowohl die Spannung als auch das Bedürfnis nach synchronisierter Zusammenarbeit steigern kann.

Auch Tiny Room Stories liefert wertvolle Impulse für Connecting-Minds, insbesondere durch seine klar strukturierte Steuerung, sowie die Verknüpfung aus über- und untergeordneter Rätselkomponenten. Die Idee, dass eine Spielhandlung des einen Spielers zu einer neuen Möglichkeit für den anderen führt, unterstützt die Vision von einem dynamischen, bidirektionalen Spielfluss. Das Prinzip des \say{wechselseitigen Fortschritts} lässt sich als zentrales Designziel ableiten.

Die Analyse von Myrmidon zeigt, dass bestimmte Parallelen zu Connecting-Minds bereits bestehen. Darunter zählt das wechselseitige Freischalten und Nutzen von Gegenständen zur Überwindung gemeinsamer Hindernisse. Zugleich treten zentrale Unterschiede hervor, etwa in der Sichtbarkeit des anderen Spielers in der Spielwelt. Während in Myrmidon die Puppe in der Kulisse des Animators visuell präsent ist, wird in Connecting-Minds gezielt darauf gesetzt, dass die Position des Players nur der Player kennt. Dadurch soll die verbale Kommunikation und räumliche Beschreibung gefördert werden.
Darüber hinaus offenbart Myrmidon eine einseitige Aufgabenverteilung. Der Animator dient als Unterstützer für seine Stop-Motion-Puppe. Das lösen von eigenständigen oder durch die Puppe unterstützende Rätsel oder Hindernisse wurden nicht realisiert. Diese Arbeit hingegen zielt genau darauf ab, dass der Watcher an gewählten Abschnitten Rätsel lösen muss. Unterstützt wird er dafür vom Player. Dies soll eine stärker balancierte Kooperationsstruktur ermöglichen. Dieser wechselseitige Rollenwechsel ist in den analysierten Vergleichsspielen bislang kaum ausgeprägt und stellt damit ein potenzielles Alleinstellungsmerkmal von Connecting-Minds dar.

\section{Methodendiskussion}\label{sec:analysis-discussion}
Im letzten Schritt in diesem Kapitel erfolgt eine kritische Diskussion über das gewählte Vorgehen und die gewählte Stichprobe, sowie die erlangten Ergebnisse aus der Analyse.

Zunächst wurde für diese Arbeit kein standardisiertes Vorgehen zur Analyse der Spiele angewandt. Die Analyse wurde aufgrund des Wunsches nach zielgerichteter Informationssammlung durchgeführt. Ein standardisiertes Vorgehen, wie dem \ac{MDA}-Framework (vgl. \citealp{hunicke_mda_2004}) oder anhand der \ac{CMP} (vgl. \citealp{seif_el-nasr_understanding_2010}), hätte vermutlich zu einem ähnlichen Ergebnis führen können, wäre in seiner Durchführung jedoch umfangreicher gewesen.
Außerdem würden die Ergebnisse der gewählten Methoden ausschließlich subjektive Einschätzungen und Einteilungen darstellen und machen dies auch.

Ein Aspekt, der in der Analyse der Spiele fehlt, ist die statistische Bestimmung von Schwierigkeitsgraden, bspw. die der enthaltenen Rätsel. Hierbei wurde keine Metrik gefunden, anhand bestimmt werden kann, wie einfach oder schwer manche Abschnitte sind. Entweder weil sie ein komplexes Pattern-Rekognition voraussetzen oder einfach nur bestimmte Elemente im Gedächtnis behalten werden müssen. Eine solche Metrik kann auch bei der Konzeption und Entwicklung eigener Rätsel von Vorteil sein, um etwa abschätzen zu können, bei welchen Rätseln mehrere Lösungshinweise eingebaut werden müssen.

Außerdem wurden gezielt einige Spiele aus Kapitel \textit{\ref{sec:sota}} von der Analyse ausgeschlossen, da sie entweder in ihrer Spielmechanik zu stark vom Konzept von Connecting-Minds abweichen (bspw. die Titel von Hazelight Studios oder Keep Talking and Nobody Explodes), oder weil die kooperativen Aspekte in ihnen zu plötzlich oder unerwartet auftreten, wie etwa in The Past Within. Diese Entscheidung war im Kontext der Analyse sinnvoll, da sie es ermöglichten, sich auf die Spiele zu konzentrieren, die hinsichtlich der Struktur und dem Rätseldesign eine größere Vergleichbarkeit zum geplanten Prototyp bieten.

Gerade im Hinblick auf die Forschungsfrage (\say{Welche spezifischen Eigenschaften muss eine solche Umgebung aufweisen und welche Kommunikationsparameter werden dabei angesprochen?}), welche Eigenschaften eingebaut werden müssen, war die Auswahl der analysierten Spiele hilfreich. Durch die gezielte Reduktion konnte ein klarer Fokus auf relevante Vergleichsparameter gelegt werden, ohne dass die Analyse durch eine zu große Anzahl an Beispielen überladen worden wäre. Eine breitere Auswahl an Spielen hätte potenziell zu einer Vielzahl zusätzlicher Aspekte geführt, die den analytischen Rahmen gesprengt oder den Fokus der Untersuchung verwässert hätte.
Nichtsdestotrotz wäre es denkbar gewesen, aus den ausgeschlossenen Titeln weitere Erkenntnisse für die konkrete Ausgestaltung einzelner Rätsel zu gewinnen. Dabei wären die Länge, Komplexität, Inszenierung oder Integration in die Spielwelt, die spannenden Bezugspunkte gewesen. 

Die Forschungsfrage (\say{Welche spezifischen Eigenschaften muss eine solche Umgebung aufweisen und welche Kommunikationsparameter werden dabei angesprochen?}) konnte, ergänzend zu den bereits in Kapitel \textit{\ref{sec:important-topics}} erarbeiteten theoretischen Grundlagen, durch die beobachteten Implementierungen in den analysierten Spielen weiter beantwortet werden. Die identifizierten Gestaltungsprinzipien und Mechanismen liefern dabei ein  konzeptionelles Gerüst, an dem sich die Entwicklung des Prototyps orientieren kann.