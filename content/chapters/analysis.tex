\chapter{Analyse von artverwandten Spielen zur Konzeptentwicklung}
\say{Connecting-Minds} besitzt ein grundlegendes Spielkonzept, das durch die Analyse von artverwandten Spielen aus dem Kapitel \emph{\nameref{sec:sota}} und weiteren Spiele, die durch ihr Spieldesign als Vorlage für die Anwendungen von \say{Connecting-Minds} als Vorlage dienen können, erweitert werden soll.

Außerdem soll durch durch die Ergebnisse der Analyse die Forschungsfrage \emph{\say{Welche spezifischen Eigenschaften muss eine solche Umgebung aufweisen und welche Kommunikationsparameter werden dabei angesprochen?}} beantwortet werden können.

\section{Methodik}
Zur systematischen Untersuchung der artverwandten Spiele wurde ein mehrstufiges, eigens entwickeltes Analyseraster verwendet. Dieses gleider sich in folgende Abschnitte:

\subsection{Visuelle Analyse}
Im visuellen Design wurde zunächst der Fokus auf sichtbare Rätsel- und Hinweiselemente der Spiele gelegt. Dabei wurden diese jeweils im Bild markiert. Die Markierungen dienten der späteren Auswertung, welche Aspekte dabei auffielen oder welche Arten von Rätseldesign genutzt wurden. Zusätzliche wurden so auch einzelnen Interaktionsflüsse erkannt und aufgereiht.

\subsection{Erstellung eines Diagramms zur Rätselstruktur}
Im zweiten Schritt nach der visuellen Analyse wurden für bestimmte Abschnitte oder das ganze Spiel UML-Ablauf Diagramme angelegt, die den Aufbau und die Verschachtlungen der Rätselelemente aufzeigen sollen. Hierbei wurde sich an der Arbeit von \cite{tim_schafer_grim_1996} orientiert.

\subsection{Deskriptive Übertragung}
Im dritten Schritt wurden Erkenntnisse zum Rätseldesign aus Schritt 1 (Visuelle Analyse) und Schritt 2 (Erstellung des Diagramms) zusammengefasst.

\subsection{Schlussfolgerung}
Im letzten Schritt wurden die Beobachtungen aus den vorangegangenen Schritten in Bezug auf die Konzeptentwicklung von \say{Connecting-Minds} gesetzt und Rückschlüsse in das eigene Konzept eingearbeitet.

\section{Ergebnisse der Analysen}