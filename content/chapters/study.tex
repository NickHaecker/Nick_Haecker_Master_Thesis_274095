\chapter{Evaluation des Prototyps sowie der Wirkung des Prototyps auf das Kommunikationsverhalten der Probanden}

Dieser Abschnitt behandelt das Testen der umgesetzten Anwendung sowie das Messen des Effektes, den die Anwendung haben soll. Er zeigt auf, an welchen Elementen des Prototyps Weiterentwicklungen von Nöten sind und gibt erste Ergebnisse darauf, wie seine Wirkung ist.

% Die beiden Studienschwerpunkte 


\section{Methodik}
% Die Evaluationsphase zielt primär darauf ab in der Kommunikationsforschung von asymmetrischen-Multiplayer Spielen neue Erkenntnisse zu liefern. Der gewählte Ablauf der primären Forschungsstudie erfolgt innerhalb eines kleines Experimentes, bei dem die Probanden zunächst nicht den Zweck der Studie erfahren. DIe Ergebnisse dieses Experiments werden auf quantitative Weise erfasst.
% % Die gewählte Methodik der Bewertung des Einflusses auf die Probanden ist dabei diejenige, die bei den zugrunde Legenden Arbeiten für ihre quantitativen Ergebnisse verwendet wurde. 

% Während der Kommunikationseinfluss im Vordergrund steht, wird ebenfalls auch die entwickelte Anwendung geprüft. Bei der hierfür gewählten Forschungsmethode wurden quantitative als auch qualitative Daten gesammelt. Diese dienen dazu einen Kenntnis stand darüber zu erhalten, an welchen Aspekten die Anwendung weiterentwickelt werden muss.
% Die Evaluation der Anwendung erfolgt über standardisierte Fragebögen, die in vergleichbaren Studien ebenfalls auf diese Weise verwendet wurden. Zusätzlich wurden über einen anderen Fragebogen qualitative Ergebnisse eingeholt.

Dieser Abschnitt beschreibt die methodische Herangehensweise zur Evaluation der entwickelten Anwendung im Kontext der Kommunikationsforschung bei asymmetrischen-Multiplayer Spielen. Ziel war es, sowohl die kommunikative Wirkung der Anwendung als auch ihre funktionale und gestalterische Tauglichkeit zu Untersuchung. Die gewählte Vorgehensweise kombiniert qualitative und quantitative Methoden innerhalb eines experimentellen Studiensettings, um ein möglichst umfassendes Bild der Nutzung und der Interaktionen zu gewinnen.

\subsection{Forschungsdesign}
Zur Untersuchung der Forschungsfragen \say{Welche Verbesserungen in der Kommunikation zwischen den Anwendern können durch ein asymmetrisches Multiplayer-Spiel mit zwei verschiedenen Spielerklassen beobachtet werden?} und \say{Wie stehen die Nutzer zu einem spielerischen Ansatz und zur Verbesserung der Kommunikation, insbesondere auch im Umgang mit Fremden?} wurde ein praxisorientiertes, experimentelles Forschungsdesign gewählt. Im Zentrum steht das asymmetrische-Multiplayer-Szenario, in dem jeweils zwei Personen unterschiedliche Rollen mit ungleich verteilten Informationen übernehmen. Diese Konstellation ermöglicht es, die Wirkung der Anwendung auf kooperative Kommunikationsprozesse zu analysieren.

Das Design sieht in der Erhebung sowohl quantitative als auch qualitative vor. 

\subsection{Erhebnungsintrumente}

% \section{Zielsetzung}

% \section{Planung und Durchführung}

% \section{Auswertung der Tests}

