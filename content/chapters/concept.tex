\chapter{Konzeption und Aufbau des Prototyps}
% In diesem Kapitel wird die Konzeption des Prototyps vorgestellt.
% % Sie enthält zunächst das methodische Vorgehen des Abschnitts, im Anschluss folgt eine Analyse der Spiele, die im Fokus der Arbeit stehen und aus welchen Aspekte in die Konzeption eingeflossen sind. Der Hauptteil der Konzeption umfassen das Game-Design des Prototyps.

Nachdem durch die Literaturrecherche in Kapitel \ref{sec:related-works} wichtige Funktionalitäten im Design identifiziert werden konnten und vergleichbare Spiele in Kapitel \ref{sec:analysis} auf ihr Game- und Rätseldesign analysiert wurden, kann aus den Ergebnissen \say{Connecting-Minds} nun vollständig konzipiert werden. \cite{krekhov_puzzles_2021} beschreibt eine Taxonomie für analoge und digitale Escape-Room Spiele, die ebenfalls in die Konzeption des Spiels hineinfließen. Außerdem werden sinnvolle Design-Patterns für kooperative Spiele aus den Arbeiten von \cite{guimaraes_rocha_game_2008} und \cite{seif_el-nasr_understanding_2010} ausgewählt und integriert.

Die Konzeption dieses Spiels folgt einem methodischen Vorgehen: Zunächst wird die übergeordnete Designvision und Zielsetzung erläutert. Anschließend dient das \ac{MDA}-Framework (vgl. \cite{hunicke_mda_2004}) als zentrales Analyse- und Strukturierungsmittel, um das Spielerlebnis gezielt zu gestalten. Auf Basis des \ac{MDA}-Frameworks werden die grundlegenden Spielmechaniken, Rollenverteilungen und dynamischen Abläufe beschrieben.

\section{Designziele und Zielgruppe}
\say{Connecting-Minds} verfolgt das Ziel, kooperative Kommunikation unter ungleichen Perspektiven in einem Escape-Room-ähnlichen Szenario zu verbessern. Dabei existieren zwei verschiedene Rollen - Player und Watcher - die gemeinsam die Rätsel lösen und Hindernisse der Spielwelt überwinden müssen. Die Wahrnehmung der Spielwelt und Handlungsmöglichkeiten ist in beiden Anwendungen unterschiedlich, gleichzeitig ergänzt sie sich aber. 

Ziel ist es, durch asymmetrische Informationsverteilung eine Spannung zwischen Orientierung und Vertrauen aufzubauen und gleichzeitig den gemeinsamen Fortschritt in den Mittelpunkt zu stellen.

Die Zielgruppe bleibt dabei identisch zu der, die in der vorangegangenen Ausarbeitung im Modul \emph{Interaktionsdesign} entstanden ist.

Zunächst gibt es da den 19 Jahre alten Steve Works, der Medieninformatik-Student im ersten Semester ist. Er strebt nicht nur nach akademischen Erfolg, sondern auch nach sozialer Interaktion. Er plant gerne Spieleabende und Partys um Kontakte zu knüpfen und um den Spaß am Studieren zu betonen.

Uwe Kaufmann, 64 Jahre alt, ist ein erfahrenere Projektleiter und steht vor der Herausforderung ein neues Team zu formen. Sein Ziel ist es, durch Teambuilding und Motivation eine effektive Zusammenarbeit zu formen.

Anja Gayms, 31 Jahre alt, ist eine introvertierte Zahnarzthelferin und such in \say{Connecting-Minds} nach einer geistigen Herausforderung und einer Möglichkeit, ihre Freundschaften zu stärken.

Die 3 Personae sind in ihrer Gesamtheit in Anhang \ref{} verfügbar.


\section{Narratives und funktionales Grundgerüst}
[Text zur Spielwelt, wo sich Schauplätze überall befinden ausdenken]


[Text nochmal kürzen]
Die beiden Rollen des Spiels werden Player und Watcher genannt. Der Player bewegt sich mit einer Isometrischen Perspektive durch die Spielwelt und kann die Ansicht bei Bedarf auch in die Egoperspektive wechseln. Er befindet sich dabei innerhalb der Spielwelt und interagiert von dort mit ihr.
Der Watcher befindet sich ebenfalls in einer Isometrischen Ansicht, besitzt allerdings keinen eigenen Avatar in der Spielwelt. Er befindet sich außerhalb der Spielwelt, kann aber dennoch mit der Spielwelt interagieren. Der Watcher kann die Spielwelt in \ac{AR} vor sich platzieren um so um die Spielwelt herum zu gehen, oder sich einzelne Elemente näher anschauen.

Das Narrativ der Handlung und der Geschichte werden durch Textuelle Hinweise und Geschichten sowie dem Aufbau der Umgebung erzählt. 

Funktional basiert das Spiel auf dem lösen von Rätseln und beseitigen von Hindernissen, die nur durch die Kombination beider Rollen lösbar sind. Das Genre des Spiels ist demnach ein kooperatives Adventure-Spiel das in einem \ac{Sci-Fi}-Setting spielt.

\section{Spielkonzeption mithilfe des MDA-Frameworks}



% \subsection{Inhalte aus der Literaturrecherche}
% \subsection{Inhalte aus der Analyse der Spiele}
% \subsection{Inhalt aus der Taxonomie}



\subsection{Aesthetics}
\subsection{Dynamics}
\subsection{Mechanics}


\section{Rollenspezifisches Design}

% \section{Genre}

% \section{Spielmechanik}

% \section{Spielablauf}

% \subsection{Spielablauf des Spiels}

% \subsection{Levelablauf}

% \section{Session}

% % \section{Belohnungen}

% \section{Spielerrollen}

% \subsection{Player}

% % \subsubsection{Interaktion mit Gegenständen}

% % \subsubsection{Tragen von Gegenständen}

% \subsection{Watcher}

% % \subsubsection{Platzieren von Gegenständen}

% % \subsubsection{Entfernen von Gegenständen}

% % \subsubsection{Previewen von Gegenständen}

% \section{Gegenstände}

% \subsection{Leichte Gegenstände}

% \subsection{Schwere Gegenstände}

% \subsection{Hinweise}

% \section{Leveldesign}

% \subsection{Gegenstände}

% \subsection{Hinweise}

% \subsection{Hindernisse}

% \section{Informationen für den Spieler}

% \section{Sounddesign}

% \chapter{Visuelles Design des Prototyps}

% \section{Moodboard}

% \section{Art-Stil}

% \section{Avatar des Players}

% \section{User Interface}

% \section{Führung durch das Level}

% \section{Gegenstände}

% \section{Menü}

% \section{Leveldesign}