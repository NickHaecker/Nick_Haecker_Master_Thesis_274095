\chapter{Einleitung}
% Einleitung über Themen von multiplayern, zusammen spielen, vl corona noch, dann darüber dass es wenig asymmetrische multiplayer gibt, irgendwie steam lib durchgehen oder andere Listen noch, verweise auf crossplattform noch geben da es ähnlich wäre.

% Einsamkeitsstudien von corona zeigen und auflisten von studien or Papern bei denen es darum geht, dass die Spiele spaßig sind und kommunikationsfördernd und social engagement steigernd sind um sowas wie corona entgegenzuwirken.

% Kurzer Überblick über den Beginn der Arbeit; enthält den Interaktionsdesignworkshop
% Titel des Projektes

% [TODO: Wissenschaftliche Einleitung zu Themen wie Corona, bestehenden Online Games usw geben]
Während der Covid-19-Pandemie führten Maßnahmen wie Lockdowns und soziale Distanzierung zu massiver psychischer Belastung, insbesondere durch gesteigerte Einsamkeit und eingeschränkte soziale Interaktionen. Studien zeigen, dass Online-Gaming als Bewältigungsstrategie diente und dabei helfen konnte, Stress, Angst, Depressionen und Einsamkeit zu reduzieren. Insbesondere spielerische soziale Interaktionen dienten dabei als Stütze (vgl. \cite{lewinson_gaming_2023}). So ergab eine systematische Übersichtarbeit, dass Online-Gaming jungen Erwachsenen weltweit als wichtiger Faktor zur sozialen Vernetzung diente, gleichzeitig aber auch Beziehungen und Empathie bei intensiver Nutzung beeinträchtigen kann (vgl. \cite{park_global_2025}). 

Auch heute leiden noch viele junge Erwachsene unter den Folgen der Lockdowns und genereller Einsamkeit (vgl. \cite{peer_junge_2024}, \cite{noauthor_einsamkeitsreport_2024}). Darüber hinaus zeigt sich, dass je weniger Kommunikation zwischen Individuen besteht (sei es persönlich oder per Telefon) die Einsamkeit von jungen Erwachsenen erhöht ist (vgl. \cite[S. 3]{sakurai_who_2021}). Daraus lässt sich ableiten, dass soziale Kommunikation nicht nur passiv entstehen sollte, sondern gezielt angestoßen, strukturell ermöglicht und von Spiel- und Interaktionssystemen eingefordert werden kann, um soziale Isolation präventiv zu bekämpfen. 

Ein erstes Konzept zur Auseinandersetzung mit diesen Problematiken wurde im Rahmen der Veranstaltung \emph{Interaktionsdesign} im ersten Mastersemester der Studiengänge\emph{ \ac{MIM}} und \emph{\ac{DIM}} im Wintersemester 2023/2024 entwickelt und umgesetzt. Hinzu kommt der Aspekt, dass Co-Lokalitiertes Spielen zur Verbesserung der Kommunikation beitragen können und daher ein Teilaspekt dieser Anwendung ist (vgl. \cite[S. 9]{goddard_designing_2016}).

Aufgrund der interessanten Rätselmechanik, des positiven Feedbacks bei den in der Projektausarbeitung enthaltenen Probandentests sowie des spannenden Forschungsfeldes wurde beschlossen, dieses Projekt weiterzuführen und die konzeptionellen Grundideen in eine Umsetzung zu überführen.

Der Titel des Projektes lautet \say{\emph{Connecting-Minds}}.

\section{Motivation und Aufgabenstellung}
Das Ziel dieser Masterarbeit ist es, den bisherigen Prototyp, der ein Mindestmaß an Funktionen des Konzepts enthielt, technisch neu zu entwickeln. Zusätzlich dazu soll der Prototyp als Versuchsumgebung dienen, um Effekte auf das Kommunikationsverhalten der Spieler zu erforschen. 

Die folgenden Forschungsfragen bilden das Grundgerüst dieser Abschlussarbeit:


\begin{itemize}
    \item \textbf{Kann eine spielbasierte Umgebung für die Untersuchung und Verbesserung von Kommunikation zwischen zwei oder mehreren Personen realisiert werden?}
    \item \textbf{Welche spezifischen Eigenschaften muss eine solche Umgebung aufweisen und welche Kommunikationsparameter werden dabei angesprochen?}
    \item \textbf{Welche Verbesserungen in der Kommunikation zwischen den Anwendern können durch ein asymmetrisches Multiplayer-Spiel mit zwei verschiedenen Spielerklassen beobachtet werden?}
    \item \textbf{Welche Unterschiede können in der Art das Kommunikationsverhalten bei der Verwendung von zwei unterschiedlichen Anwendungen (AR und 3D) (festgestellt/beobachtet) werden}
    \item \textbf{Wie stehen die Nutzer zu einem spielerischen Ansatz und zur Verbesserung der Kommunikation, insbesondere auch im Umgang mit Fremden?}
\end{itemize}

\section{Struktur der Arbeit}
Diese Arbeitet startet mit der Einführung in die theoretischen Grundlagen in \say{Kapitel \ref{sec:basics}: \nameref{sec:basics}}, die für diese relevant sind. Im Anschluss folgt in \say{Kapitel \ref{sec:related-works}: \nameref{sec:related-works}} der bisherige Stand der Forschung zur Verbesserung der Kommunikation und sie erreicht werden kann. Außerdem gibt sie eine Zusammenfassung der wichtigen Begrifflichkeiten dieser Arbeit und präsentiert den eigenen Beitrag zur Forschung. In \say{Kapitel \ref{sec:sota}: \nameref{sec:sota}}) werden artverwandte Spiele vorgestellt, die Konzepte aus dem Kapitel \ref{sec:related-works} enthalten und für die Öffentlichkeit zugänglich und spielbar sind. \say{Kapitel \ref{sec:analysis}: \nameref{sec:analysis}} analysiert die relevantesten Spiele aus Kapitel \ref{sec:sota} und stellt Kernaspekte heraus, die für die eigene Konzeptentwicklung wichtig sind.
% [TODO: Zusammenfassen in welchen Kapitel sich was befindet]