\chapter{Theoretische Grundlagen}

In diesem Kapitel werden die für den Forschungshintergrund und für die Entwicklung des Prototyps wichtigen theoretischen Grundlagen vorgestellt.
Zunächst werden verschiedene Theorien zu Kommunikationsmodellen vorgestellt, auf deren Basis das Modell für diese Anwendung zugrunde liegt.
Im Anschluss daran werden die in der Ludologie beschriebenen Akteure vorgestellt, welche sich in den Probanden dieser Masterarbeitsstudie wiederfinden werden. Allgemein ist bekannt, dass Video- und Computerspiele drei verschiedene Modi haben können: Singleplayer, Multiplayer und Mischformen. Da für den Zweck dieser Studie ein Multiplayer-Spiel konzipiert und umgesetzt wurde, werden im weiteren Verlauf verschiedene Kategorien von Multiplayer-Spielen vorgestellt. Außerdem werden die damit einhergehenden Netzwerkinfrastrukturen vorgestellt, die relevant sind und für die weitere Entwicklung relevant sein könnten.

\section{Kommunikationsmodelle}
% In der Kommunikationswissenschaft wird die Kommunikation in 2 Arten unterteilt, die Massen- und Individualkommunikation.

\subsection{Nach Schulz von Thun}
\subsection{Nach Wazlawik}
\subsection{Nach Rogers}

[Könnte besser in die Einleitung passen
\section{Spiele als soziales Medium}
\cite{depping_trust_2016}
\cite{gerling_designing_2014}
\cite{ducheneaut_alone_2006}
]

\section{Spielertypen}

\subsection{Nach Bartle}
\cite{bartle_hearts_nodate}

[erwähnen wurde aber rausgelassen, man kann erwähnen, dass es noch weitere klassifizierungen gibt
\subsection{Das BrainHex-Model}
\cite{nacke_brainhex_2014}
]

[kommt zu wichtige Begriffe
\section{Kooperative Gamedesign Pattern}
\subsection{Was sind Game Pattern}
\cite{bjork_patterns_2005}

\subsection{Complementarity}

\subsection{Synergies}

\subsection{Abilities}

\subsection{Shared Goals}

\subsection{Synergies between goals}

\subsection{Special Rules for Player of the same Team}

\subsection{Camera Setting}

\subsection{Interacting with the same object}

\subsection{Shared puzzle}

\subsection{Shared characters}

\subsection{Special characters targetting lone wolf}

\subsection{Vocalization}

\subsection{Limited ressources}

\subsection{Einflussnahme}
\cite{emmerich_impact_2017}

]
\section{Multiplayerspiele}

\subsection{Klassifizierungen}

\subsubsection{Synchrone Multiplayer}

\subsubsection{Asynchrone Multiplayer}

\subsubsection{Symmetrische Multiplayer}

\subsubsection{Asymmetrische Multiplayer}

\subsection{Artverwandte Beispiele}
Hier kommen die analysierten Spiele rein, also die Auflistung der Spiele, die ich mir im Zuge angesehen habe

\section{Netzwerkinfrastrukturen}

enthält eine Liste von Möglichkeiten auf welcher Grundlage verschiedene Multiplayer Anwendungen gebaut werden können


\chapter{Verwandte Arbeiten}

\cite{harris_asymmetry_2019}
\cite{sajjadi_maze_2014}

hier würden Paper reinkommen die asymmetrische Multiplayer gemacht haben, welche aspekte da mitreinspielen, da kommen dann auch die wichtigen Begriffe dazu mitrein. Auch bereits umgesetzt asymetrische VR Spiele?


Auch Anna Lotz´ Thesis wäre hier relevant


\section{Wichtige Begriffe}

\subsection{Interdependence}
\cite{harris_leveraging_2016}
\cite{depping_cooperation_2017}

\subsection{Degrees of Interdependence}
\cite{beznosyk_effect_2012}

\subsection{Soziale Präsenz}

Vertrauen gibts in dem Kontext auch und wie man dan über Spiele aufbaut

\section{Untersuchungsschwerpunkte}