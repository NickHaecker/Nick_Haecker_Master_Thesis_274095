\chapter{Diskussion der Ergebnisse}\label{sec:discussion}

Ziel dieser Arbeit war es, das Konzept sowie die technische Umsetzung der vorausgegangenen Untersuchung weiterzuentwickeln und für die Anwendung der Watcher zwei unterschiedliche Versionen zu konzipieren. Eine dieser Versionen sollte in einer \ac{AR}-Umgebung spielbar sein, während die zweite als \ac{RTS}-ähnliche Anwendung für das Smartphone realisiert werden sollte. Anhand dieser beiden Formate sollten Unterschiede im Kommunikationsverhalten der Probanden, insbesondere aufseiten der Watcher und deren Auswirkung auf die Player, beobachtet und analysiert werden. Darüber hinaus sollte evaluiert werden, welche der beiden Anwendungen ein höheres Maß an Immersion für die Watcher erzeugt, um dieses Erleben perspektivisch stärker an jenes der Player anzugleichen.

Da die angestrebte \ac{AR}-Integration nicht erfolgreich umgesetzt werden konnte, wurde die dadurch frei gewordene Zeit gezielt in eine umfassendere Literaturrecherche verwandter Arbeiten sowie in die Auseinandersetzung mit geeigneten Auswertungsmethoden investiert. Dies ermöglichte einen fundierten Überblick über den aktuellen Forschungsstand und half dabei, geeignete Formen der Analyse zu identifizieren.

Ein zentraler Bestandteil der Arbeit war die Konzeption und Entwicklung eines asymmetrischen Multiplayer-Settings, in dem zwei unterschiedliche Spielerrollen gemeinsam ein übergeordnetes Ziel verfolgen. Zur sinnvollen Ausgestaltung dieser Rollen musste zunächst analysiert werden, in welchen Bereichen sich ihre Aufgaben unterscheiden. Im Rahmen dieser Analyse wurde das Konzept der Interdependenz als zentrales Gestaltungsprinzip identifiziert und übernommen. Auf dieser Grundlage konnten Spielmechaniken und Rätsel konzipiert werden, um eine enge Zusammenarbeit der beiden Rollen zu fördern.

Die Konzeption erforderte die Entwicklung von insgesamt drei Anwendungen. Die Netzwerkstruktur konnte auf der bestehenden Serverinfrastruktur der vorangegangenen Arbeit aufbauen, wodurch der Entwicklungsaufwand in diesem Bereich vergleichsweise gering blieb. Deutlich herausfordernder gestaltete sich hingegen die vollständige Neuentwicklung der Player- und Watcher-Anwendungen. Diese mussten über eine funktionierende Netzwerkkommunikation verfügen, Zugriff auf alle relevanten Komponenten und \ac{3D}-Objekte ermöglichen, sowie eine stabile Integration in Unity gewährleisten. Rückblickend wäre es sinnvoll gewesen, den bereits existierenden Prototyp weiterzuentwickeln, um Entwicklungszeit einzusparen und gleichzeitig die Qualität der Spielweltumsetzung zu erhöhen.

Bei der Entwicklung des Prototyps wurden zunächst die grundlegenden Kernfunktionen der Anwendungen, das Platzieren, Entfernen und Previewen von Gegenständen, entwickelt. Auf dieser Basis wurden, gestützt durch erste Spielanalysen, die Szenarien für das Tutorial umgesetzt. Rückblickend erwies sich diese Reihenfolge jedoch als ungünstig, da die \ac{AR}-Funktionalität nicht rechtzeitig funktional implementiert werden konnte. Eine frühzeitige technische Prüfung der Kompatibilität verschiedener Version der ARFoundation und Unity-Editoren hätte diese Problematik möglicherweise verhindern können. Ebenso hätte der Entwicklungsprozess besser auf die spezifischen Anforderungen der \ac{AR}-Technologien abgestemmt werden sollen. Die Entscheidung, am bestehenden Code festzuhalten, schränkte die Flexibilität erheblich ein und verhinderte ein technisch notwendiges Downgrade der \ac{AR}-Komponenten. 

Ein zentrales Nebenziel der Arbeit war die Optimierung der Systemperformance. Hierzu wurden Low-Poly-Modelle, Lightmaps und Occlusion Culling eingesetzt. Diese Maßnahmen verbesserte die grafische Darstellung und reduzierte die Auslastung der Game-Engine spürbar. Während der Tests traten keine Performanzprobleme auf, was die getroffenen technischen Entscheidungen grundsätzlich bestätigte, auch wenn retrospektiv nicht abschließend überprüft werden konnte, ob das Occlusion Culling in allen Build-Versionen korrekt aktiviert war.

Die Analyse vergleichbarer Spiele ermöglichte ein differenziertes Verständnis bestehender Spielmechaniken und diente frühzeitig als Orientierungshilfe zur Entwicklung potenzieller Alleinstellungsmerkmale. Auch die Auseinandersetzung mit dem Rätseldesign anderer Spiele lieferte wertvolle Impulse für die Konzeption eigener Aufgaben innerhalb der Spielumgebung.

Die Rekrutierung der Probanden war stark eingeschränkt. Es wurde ausschließlich Studieren der ehemaligen Fakultät \ac{DM} in die Untersuchung einbezogen. Eine Ausweitung auf Mitarbeiter, andere Fakultäten oder externe Einrichtungen (z. B. den Impact Hub Stuttgart) wäre zwar prinzipiell möglich und inhaltlich gewinnbringend gewesen, konnte jedoch aus zeitlichen und organisatorischen Gründen nicht realisiert werden. Auch die Möglichkeit auf Pflichtprobandentestungen am Standort Schwenningen zurückzugreifen, wurde aus Unkenntnis nicht genutzt. Dadurch blieb es bei den insgesamt sieben Testdurchläufen, was die Generalisierbarkeit der Ergebnisse deutlich einschränkt.

Ein zentrales Hindernis während der Testvorbereitungen stellte die technische Infrastruktur dar. Sowohl die Datenbank als auch der Server wurden lokal auf der eigenen Hardware betrieben. Ein zentrales Hosting hätte es ermöglicht, vorbereitete Builds deutlich einfacher in den Testraum zu überführen. Auch die Verwaltung der IP-Adressen hätte sich auf die Weise effizienter gestalten lassen. Die damit verbundenen technischen und logistischen Herausforderungen konnten jedoch durch hohen persönlichen Einsatz im Vorfeld der Tests erfolgreich bewältigt werden.

Darüber hinaus wären mehrere Testdurchläufe mit Probanden erforderlich gewesen, um die Gebrauchstauglichkeit der Anwendungen gezielt zu optimieren. Auf diese Weise hätten sich die Teilnehmer während der Tests stärker auf das Spielgeschehen konzentrieren können, anstatt durch bestehende gebrauchstauglichen Mängel in ihrer Interaktion und Kommunikation beeinträchtigt zu werden.

Die durchgeführten Probandentests lieferten aufschlussreiche Einblicke in das Kommunikationsverhalten der Teilnehmer. Sowohl im Vor- als auch im Nachtest zeigte sich ein hoher Anteil an \ac{CF}, dessen Anteil über beide Testzeitpunkte hinweg nahezu konstant blieb. Erste Tendenzen zur Signifikanz zeigten sich bei der Reduktion der Anzahl und Dauer von Gesprächspausen sowie bei der Zunahme initiativer Kommunikationsbeiträge seitens der Watcher-Probanden. Auch die subjektive Einschätzung der Teilnehmer deuten darauf hin, dass das Spiel die Kommunikation, insbesondere im Kontakt mit unbekannten Personen, erleichtert hat.

Insgesamt blieben die Ergebnisse jedoch hinter den ursprünglichen Erwartungen zurück. Es wurde angenommen, dass der entwickelte Prototyp, insbesondere aufgrund der asymmetrischen Rollenverteilung, einen deutlich stärkeren Einfluss auf das Kommunikationsverhalten ausüben würde. Erwartet wurden unter anderem klarere Veränderungen im Wordcount sowie ein signifikanter Anstieg kollaborativer Gesprächsanteile (\ac{CF}). Diese blieben jedoch, wenn auch auf einem vergleichsweise hohen Niveau, weitgehend konstant.

Im Vergleich zu den Arbeiten von \citeauthor{nasir_effect_2015}, in denen ein gezieltes \say{Icebreaking}-\ac{RPG}-Spiel zur Förderung einer offenen Kommunikationsdynamik in einer Testgruppe eingesetzt und mit einer Kontrollgruppe verglichen wurde, unterscheidet sich das Studiendesign der vorliegenden Arbeit wesentlich. Während in der Studie von \citeauthor{nasir_effect_2015} eine Brainstorming-Aufgabe im Zentrum stand, konzentrierten sich die durchgeführten Vor- und Nachtests dieser Arbeit auf die strukturiere Lösung einer klar definierten Aufgabe. Dieses aufgabenorientierte Szenario könnte dazu geführt haben, dass sich die Kommunikation der Probanden stärker auf operative Aspekte beschränkte und weniger Raum für spontane oder frei verlaufende Dialoge ließ. Außerdem bot die Aufgabe Möglichkeiten der non-verbalen Kommunikation, welche über die angewandte Methodik nicht gemessen werden konnte.

Aus retrospektiver Perspektive wäre es sinnvoll gewesen, für die Vor- und Nachtest ein offenes oder alltagsnäheres Szenario zu wählen, bspw. eine freie Explikationsaufgabe oder eine kooperative Gestaltungsphase ohne vorab definierter Zielvorgaben. In einem derartigen Setting hätten sich mit hoher Wahrscheinlichkeit spontanere Gesprächsdynamiken, Aushandlungsprozesse sowie Rollenerklärungen ergeben, die für die Evaluation kooperativer Kommunikation besonders aufschlussreich gewesen wären. Der in der vorliegenden Untersuchung gewählte, eher strukturierte Aufbau hingegen könnte unbeabsichtigt die kommunikativen Spielräume der Teilnehmer eingeschränkt haben. Mit Fokus auf die Bearbeitungszeit hätte ein offeneres Testsetting nicht vor Ablauf der vorgegebenen Zeit beendet werden können, wodurch allerdings eine besserer Vergleichbarkeit zwischen den Probandengruppen ermöglicht worden wäre.

Die zentralen Forschungsfragen lassen sich insgesamt wie folgt zusammenfassen. Eine spielbasierte Umgebung zur Förderung kommunikativer Prozesse ist grundsätzlich realisierbar, entfaltet ihr volles Potenzial jedoch voraussichtlich erst bei größerer Stichprobenumfängen und einem längeren, stärker ausdifferenzierten Spielinhalt. Damit eine solche Umgebung kommunikativ wirksam ist, bedarf es insbesondere einer symmetrischen Informationsverteilung, kooperativer Problemstellungen sowie immersiver Rollenverteilung. Nur unter diesen Voraussetzungen können zentrale Kommunikationsparameter, wie Gesprächsführung, Gesprächsinitiativen, Empathie, soziale Präsenz und Verständnissicherungen, gezielt adressiert und gefördert werden.

Im Rahmen der durchgeführten Studie konnten erste Hinweise auf eine Verbesserung der Kommunikation beobachtet werden, etwa in Form veränderter Gesprächsstrukturen und verkürzter Pausenzeiten, wenngleich diese unterhalb des statistischen signifikanten Niveaus bleiben. Unterschiede im Kommunikationsverhalten zwischen der \ac{AR}- und \ac{3D}-Anwendung konnten nicht analysiert werden, da die geplante \ac{AR}-Implementierung nicht funktionsfähig umgesetzt wurde und entsprechende Versuchsdurchführungen ausblieben. Dennoch zeigten sich die befragten Probanden gegenüber dem spielerischen Ansitz grundsätzlich offen. Insbesondere lies sich eine tendenzielle Reduktion von Hemmschwellen im Umgang mit unbekannten Personen feststellen.