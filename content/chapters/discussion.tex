\chapter{Diskussion der Ergebnisse} 
Ziel dieser Arbeit war es das Konzept und die Umsetzung der vorausgegangenen Arbeit zu überarbeiten und für die Anwendung des Watchers zwei unterschiedliche Versionen anzulegen. Eine der beiden Versionen sollte dabei im \ac{AR} gespielt werden können, die andere Version soll dabei eine \ac{RTS}-ähnliche Anwendung für das Smartphone werden. Anhand der beiden Anwendungsformen sollen Unterschiede im Kommunikationsverhalten der Probanden (hauptsächlich der Watcher und die Reaktionen der Player darauf) beobachtet und ausgewertet werden. Außerdem sollte anhand der Anwendungen evaluiert werden, welche den Watchern das größte Maß an Immersion bietet um dieses Empfinden in Bezug auf die Player-Anwendung anzugleichen. 
Durch die nicht erfolgreiche Umsetzung der \ac{AR}-Integration konnte die entstandene Zeit sinnvoll in die umfassende Forschung von verwandten Arbeiten und Auswertungsmethodiken investiert werden. Dies half ungemein einen Überblick über die bestehende Forschung zu gewinnen und mögliche Formen der Auswertung zu bestimmen.
%  Ziel der Arbeit nochmal vorstellen; darunter fallen auch die Einzelstudien die damit noch einhergegangen wären (a/b test) -> aufzeigen, dass durch die eingrenzung und dem nicht möglichen umsetzen einer funktionierenden ar anwendung intensiver mit der forschung beschäftigt werden konnte

Ein wesentlicher Bestandteil dieser Arbeit war die Konzeption und Entwicklung eines asymmetrischen Multiplayers, bei dem zwei verschiedene Spielerrollen miteinander an ein gemeinsames Ziel gelangen. Um diese Rollen bestmöglich zu gestalten, musste zunächst recherchiert werden, in welcher Form sich die Rollen unterschieden müssen. Durch die umfassende Recherche konnte dabei schnell das Thema \say{Interdependence} (in Deutsch: Abhängigkeiten) gefunden und als Basis des Konzeptes festgelegt werden. Aufgrund dessen konnten die Aufgaben der Spielerrollen und die Rätsel bestmöglich konzipiert und umgesetzt werden.

% wesentlicher bestandteil der anwednung waren die interdependencen zwischen den anwendungen, sowie die netzwerkinfrasdtruktur

Das Konzept setzte voraus, dass drei verschiedene Anwendungen entwickelt werden mussten. Die Entwicklung der Netzwerkinfrastruktur konnte dabei auf der Serverinfrastruktur der vorangegangenen Arbeit aufsetzen, sodass nur noch Inhalte für den neuen Prototyp entwickelt werden mussten. Unter Inhalte sind Endpunkte für die WebSocket Nachrichten und die entsprechenden Express Endpunkte für die Datenbank Kommunikation gemeint. Das bestehende Konstrukt konnte dabei einen enormen zusätzlichen Aufwand verhindern. Anders sah es bei den Anwendungen des Players und Watchers aus. Diese mussten von Grund auf neu Entwickelt werden. Dabei entstanden neue Herausforderungen die bewältigt werden mussten. Zunächst mussten die Kernelemente der Netzwerkkommunikation für beide Anwendungen eingerichtet werden, außerdem brauchten beide Anwendungen Zugang zu allen Komponenten und \ac{3D}-Objekten die verwendet werden sollten. Der die dabei entstandenen koordinativen Herausforderungen sind für die Art einer solchen Arbeit zu anspruchsvoll gewesen. Außerdem musste ein passendes und sinnvolles Szenario für den Gesamtverlauf der Handlung entworfen werden, welcher zuvor nicht existent war. Rückblickend hätte zunächst der Ansatz verfolgt werden sollen, das bestehende Fundament weiter zu entwickeln, anstatt von Grund auf neue Wege zu gehen. Auf diese Weise hätte in der anfänglichen Umsetzung der Spielwelt Zeit eingespart werden können, welche für die Ausgestaltung der Anwendungen sinnvoller investiert worden wäre.
% die fehldene Grundlage die anwendungen zu entwickelten verursachten einen hohen aufwand, da in 2/3 der anwendungen von 0 gestartet werden musste und sich die welt erst überlegt werden musste; glücklicherwiese konnte bei der entwicklung des netzwerkes auf einer sehr gut und breit ausgelegten grundlage aufgebaut werden und quasi nur inhalte entwickelt werden mussten

Zu Beginn der Umsetzung wurden zunächst die Kernfunktionen der Anwendungen entwickelt. Im Anschluss erfolgte, unterstützt durch die Ergebnisse der Spielanalysen, die Umsetzungen der Szenarien aus dem entwickelten Tutorial. Dieser Ansatz zeigte sich als Fehler, da die \ac{AR}-Funktionalitäten nicht mehr rechtzeitig fehlerfrei umgesetzt werden konnten, bevor die geplanten Nutzertests stattfinden sollten. Bevor überhaupt mit der Umsetzung begonnen wurde, hätte bereits eine Recherche stattfinden sollen, welche Version der ARFoundation fehlerfrei funktioniert, um darum die Versionen des Unity-Editors und der anderen Packages auszuwählen. Außerdem hätte die etablierte Game-Loop des Prototyps auch besser an die Gegebenheiten der \ac{AR}-Funktionalitäten angepasst werden können. Dabei geht es primär um das Platzieren der Spielwelt und wie sich die enthaltenen Komponenten in der gesamten Game-Loop verhalten. Über den in der Entwicklung gewählte Weg, war die Flexibilität des Systems nicht mehr wie gewünscht gegeben und konnte auch aus der Sorge eines Funktionsverlusts der genutzten Erweiterungen kein Major-Downgrade der ARFoundation-Version vollzogen werden.

% der Fokus erst die kernmechanik zu entwickeln und dann sich um die entwicklung der ar anwendung zu kümmern zeigte sich als ein Fehler, da so die ar anwendung mangels fehlerhafter version nicht funktionsfähig war; man hätte während der entwicklung des systems bzw vor der entwicklung darauf achten sollen, welche version für die ar integrtation die richtige gewesen wäre

Die Anwendungen des Players und Watchers sollten eine gute Performance haben. Dies war ein zentrales Nebenziel in der Entwicklung. Um eine gute Performance zu erhalten wurden sogenannte Low-Poly Modelle verwendet, welche der Engine für die Darstellung einen geringen Rechenaufwand bietet. Zudem wurden Lightmaps angelegt, welche das Berechnen von dynamischen Lichtquellen in Echtzeit auf ein Mindestmaß reduziert hat. Außerdem boten die Lightmaps ein gutes Stimmbild ab, welches auch den Probanden positiv in Erinnerung blieb. Zusätzlich dazu wurde das Occlusion-Culling von Unity aktiviert, wodurch nur für die Kamera sichtbare Objekte berechnet werden. Dies fördert das Benutzererlebnis durch eine gute Performance. Ob das Occlusion-Culling in den entsprechenden Builds aktiv war und einen Einfluss hatte, konnte rückblickend leider nicht überprüft werden. Allerdings gab es während der Probandentests keine Performance-Probleme wodurch sich der Einsatz gelohnt hat.

% Außerdemn lag ein Fokus auf der Perfomrnace der Anwendungen wodurch mit vorgeneriteren Lightmaps und dem occlusion culling gearbeit wurde; Die Leightmapsm zeigten sich in den fertigen Anwendungen als zuiemlcih stimmig und erzeugten ein gutes Feedbaxck der Teilnehmer, ob das occlusion culling aktiv war konnte leider nicht besttigt werden, allerdings war die insgesammte performance der Anwendungenm, was framerate angeht, sehr gut 

Die Analyse der artverwandten Spielen und Arbeiten gaben einen guten Überblick über bestehende Funktionalitäten und Mechaniken und gaben in der Frühphase der Konzeption gute Differenzierungsmöglichkeiten um Alleinstellungsmerkmale etablieren zu können. Gleiches gilt auch in der theoretischen Arbeit des Rätseldesigns, welches insgesamt einen guten Einblick in die Konzeption von Rätseln gibt. 

% Durch die Analyse von bestenenden Spielen konnten frühzeitig differenzioerungen in den mechaniken festgelegt werden und man sich relativ schnell in der konzeptionsphase um die entwicklung eigener rätsel konzentrieren konnte

Video- und Computerspiele werden durch nahezu alle Altersgruppen hinweg gespielt. Daher wäre es nur logisch gewesen nicht nur Studierende der ehemaligen Fakultät Digitale Medien (jetzt Fakultät 1 und 4) für die Probandentests einzuladen, sondern auch Mitarbeitenden/ Dozierende der Fakultäten. Außerdem gäbe es auch die Option Selbständige oder Angestellte des Impact-Hubs in Stuttgart zu den Probandentests einzuladen. Aufgrund des großen Aufbaus des Spiels und der umfassenden Forschungsstudie wurde sich nur auf den Aufbau an der Fakultät konzentriert. Des weiteren hätte bei anderen Fakultäten oder anderen Standorten um weitere Probanden bemüht werden können. Aufgrund der aufwändigen und Planung und Organisation wurden diese Möglichkeiten leider außer acht gelassen. Zumal es in Schwenningen offenbar Pflichtteilnahmen für Probandentests gibt, welche für den Zweck dieser Studie gerne zugunsten genommen worden wäre. So wurde leider die Chance nicht genutzt mehr als nur sieben Versuchsdurchläufe durchführen zu können.

% man traf zunächst die Wahl nur innerhalb der Hochschule nach Probanden zu fragen, primär studierende; für den Zweck der Studie hätten auch weitere personengruppen wie Dozenten und Mitarbeiter aquiriert werrden können; außerdem hätten noch weitere Probanden im Co-Working Space meiner Fiorma nach mehr Probanden gefragt werden können; so wurden es nur 14 Probanden bei 7 Versuchsdurchläufen

Der Aufbau des Probandentests mit der ausgeliehenen Hardware hätte durch ein zentrales Hosting der Datenbank und des Express Server deutlich reduziert werden können. Die einzelnen Anwendungen hätten in der Entwicklung am eigenen Arbeitsplatz vorbereitet werden können und dann als fertige Anwendung zum Testraum gebracht werden können. Da sich zwischenzeitlich nicht mehr (aus Gründen des Umfangs und der begrenzten Zeit) um ein Hosting bemüht wurde, wurde die Netzwerkinfrastruktur über eigene Hardware umgesetzt. Die Herausforderung dabei ist die, dass in den einzelnen Anwendungen jeweils die IP-Adresse des Gerätes angegeben werden muss, auf dem der Express-Server ausgeführt wird. Anfangs wurde ein TP-Link Router ausgeliehen, für den zunächst die Projekte auf den ausgeliehenen Geräten importiert und gebaut werden mussten. Ein Aufwand der durch einen eigenen Router gemindert werden könnte. Außerdem gab es im Verlauf der Vorbereitungen der Probandentests ach Herausforderungen bezüglich der Koordination der Räume und der Hardware, welche allesamt mit einem positiven Ergebnis gelöst werden konnten.

% Für den Zweck der Probandentests hätte auch ein geplantes hosting untersatützen können, da es den Aufbau verkleinert hätte und die Tests insgesamt angenehmer gestaltert hätte; allerdings wäre man so weniger flexibel, falls Probleme aufgestoßen wären

% Bei der Durchführung der Tests gabs einige orgamnisatorische Herausforderung wie Raumreservierungs überschneidungen oder ausgeliehende Hardware die nach Absprache plötzlich doch nicht mehr zur Verfügung standen, diese konnten aber geklärt werden


% insgesamt zeigte sich trotz des sehr großen Umfangs der Arbeit eine ordentliche Entwicklung eines stylistischen Prototyps mit ansprechender FOrschungsmethode und Durchführung, auch wemn die Ergebnisse der Studie nicht die gewünschten Ergebnisse liefern konnten.

