\chapter{Diskussion der Ergebnisse} 
Ziel dieser Arbeit war es, das Konzept und die Umsetzung der vorausgehenden Arbeit zu überarbeiten und für die Anwendung des Watchers zwei unterschiedliche Versionen anzulegen. Eine der beiden Versionen sollte dabei in \ac{AR} gespielt werden können, die andere Version eine \ac{RTS}-ähnliche Anwendung für das Smartphone werden. Anhand dieser beiden Anwendungsformen sollten Unterschiede im Kommunikationsverhalten der Probanden - insbesondere der Watcher und deren Auswirkungen auf die Player - beobachtet und ausgewertet werden. Außerdem sollte evaluiert werden, welche Anwendung den Watchern das größte Maß an Immersion bietet, um dieses Empfinden in Bezug auf die Player-Anwendung anzugleichen
Durch die nicht erfolgreiche Umsetzung der \ac{AR}-Integration konnte die dadurch frei gewordene Zeit sinnvoll in die umfassende Recherche verwandter Arbeiten und Auswertungsmethoden investiert werden. Dies ermöglichte es, eine umfassenden Überblick über den Forschungsstand zu erhalten und geeignete Formen der Auswertung zu identifizieren.

Ein wesentlicher Bestandteil dieser Arbeit war die Konzeption und Entwicklung eines asymmetrischen Multiplayers, bei dem zwei verschiedene Spielerrollen gemeinsam ein Ziel verfolgen. Um diese Rollen sinnvoll zu gestalten, musste zunächst erarbeitet werden, wie sich ihre Aufgabenbereiche voneinander unterschieden. Im Zuge dieser Recherche wurde das Konzept der \textit{Interdependence} identifiziert und als zentrales Gestaltungsprinzip übernommen. Dies ermöglichte eine gezielte Konzeption der Spielmechaniken und Rätsel, die die Zusammenarbeit der Rollen voraussetzen.

Das Konzept verlangte die Entwicklung dreier Anwendungen. Die Netzwerkstruktur konnte auf der Serverinfrastruktur der vorausgegangenen Arbeit aufbauen, sodass der Entwicklungsaufwand hier vergleichsweise gering war. Herausfordernder gestaltete sich hingegen die Entwicklung der Player- und Watcher-Anwendungen, die vollständig neu erstellt werden mussten. Beide benötigten eine funktionierende Netzwerkkommunikation, Zugriff auf alle relevanten Komponenten und \ac{3D}-Objekte sowie eine stabile Unity-Integration. Rückblickend hätte der bereits vorhandene Prototyp weiterentwickelt werden sollen, um Entwicklungszeit zu sparen und die Umsetzungsqualität der Spielwelt zu erhöhen.

Zunächst wurden die Kernfunktionen (Platzieren, Entfernen, Previewen von Gegenständen) der Anwendungen entwickelt. Darauf aufbauend wurden, gestützt durch erste Spielanalysen, die Szenarien für das Tutorial umgesetzt. Dieser Ablauf stellte sich im Nachhinein als ungünstig heraus, da die \ac{AR}-Funktionalität nicht rechtzeitig fertiggestellt werden konnte. Eine frühzeitige technische Recherche zur Funktionsfähigkeit der ARFoundation-Versionen und Unity-Editoren hätte diese Problematik möglicherweise verhindert. Auch der Software-Loop hätte frühzeitig besser auf die Besonderheiten der \ac{AR}-Technologien abgestimmt werdne können. Die Entscheidung, am bestehenden Code festzuhalten, führte zu einer geringeren Flexibilität und verhinderte ein notwendiges Downgrade der \ac{AR}-Komponenten.

Ein zentrales Nebenziel war die Optimierung der Performance. Hierfür wurden Low-Poly-Modelle, Lightmaps und Occlusion-Culling eingesetzt. Diese Maßnahmen verbesserten die Darstellung und entlasteten die Engine spürbar. Während der Test traten keine Performance-Probleme auf, was den gewählten technischen Maßnahmen eine gewisse Berechtigung verleiht, auch wenn retrospektiv nicht vollständig überprüft werden konnte, ob das Occlusion-Culling in allen Builds aktiv war.

Die Analyse verwandter Spiele ermöglichte ein differenziertes Verständnis für bestehende Mechaniken und bot frühzeitig Orientierung, um Alleinstellungsmerkmale zu entwickeln. Auch die Auseinandersetzung mit deren Rätseldesign lieferte wertvolle Impulse für die Konzeption der Aufgaben im Spiel.

Die Rekrutierung der Probanden war jedoch limitiert: Es wurden ausschließlich Studierende aus der ehemaligen Fakultät Digitale Medien einbezogen. Eine Ausweitung auf Mitarbeitende, andere Fakultäten oder externe Standorte (z. B. Imapct Hub Stuttgart) wäre möglich und sinnvoll gewesen, wurde aber aus zeitlichen und organisatorischen Gründen nicht realisiert. Auch die Möglichkeit, auf Pflichtprobandentestungen in Schwenningen zurückzugreifen, wurde aus Gründen des fehlenden Wissens nicht genutzt. Dadurch blieb es bei sieben Testdurchläufen, was die Generalisierbarkeit der Ergebnisse einschränkt.

Ein zentrales Hindernis während der Testvorbereitungen war die Infrastruktur. Die Datenbank und der Server liefen lokal auf eigener Hardware. Durch ein zentrales Hosting hätten vorbereitete Builds einfacher in den Testraum gebracht werden können. Auch die IP-Adressenverwaltung hätte sich dadurch vereinfachen lassen, Die Herausforderungen - sei es technischer oder logistischer Natur - konnten jedoch durch viel Einsatz im Vorfeld der Tests erfolgreich bewältigt werden.

Die durchgeführten Probandentest lieferten interessante Einblicke in das Kommunikationsverhalten der Spieler. So zeigten sich sowohl im Vortest, als auch im Nachtest große Anteile des \ac{CF}, der im Vergleich nahezu unverändert blieb. Erste Tendenzen zur Signifikanz zeigten sich sich bei der Reduzierung der Pausenanzahl und Länge, sowie der steigenden Gesprächsinitiativen der Watcher-Probanden. Auch die subjektive Einschätzung der Probanden deutet darauf hin, dass die Kommunikation durch das Spiel erleichtert wurde - insbesondere im Umgang mit unbekannten Personen. 

Insgesamt blieben die Ergebnisse jedoch hinter den ursprünglichen Erwartungen zurück. Es wurde angenommen, dass der entwickelte Prototyp, insbesondere durch seine asymmetrische Rollenverteilung, einen stärkeren Einfluss auf das Kommunikationsverhalten ausübt. Erwartet worden war bspw. eine klarere Veränderung im Wordcount oder eine deutlichere Steigung kollaborativer Gesprächsanteile (\ac{CF}). Diese blieben jedoch weitestgehend, bei einem hohen Niveau, konstant.

Im Vergleich zu den Arbeiten von Nasir, in denen ein gezieltes Icebreaking-Spiel bei einer Testgruppe eingesetzt wurde und der Effekt im Vergleich zu einer Kontrollgruppe angewandt wurde, um eine offene Kommunikationsdynamik zu erzeugen, unterscheidet sich das vorliegendes Studiendesign. Während dort der Aufbau der Spielsituation explizit um eine Brainstorm-Aufgabe handelte, lag hier der Fokus der hier durchgeführten Vor- und Nachtests auf der strukturierten Lösung einer klar definierten Aufgabe. Dieses aufgabenorientierte Szenario könnte dazu geführt haben, dass sich die Kommunikation der Probanden stärker auf operative Aspekte beschränkte und weniger Raum, für die freie, dynamische Dialogverläufe bot. 

Aus retrospektiver Sicht wäre es daher sinnvoll gewesen, ein freieres oder alltäglichere Szenario für die Vor- und Nachtests zu wählen, bspw. eine offene Explikationsaufgabe oder eine kooperative Gestaltungsphase ohne festgelegte Ziele. In einem solchen Setting hätten sich vermutlich deutlicher spontane Gesprächsdynamiken, Verhandlungen oder Rollenaushandlungen ergeben, die für die Evaluation kollaborativer Kommunikation besonders aufschlussreich gewesen wären. Der vorangegangene Aufbau hingeben könnte unbeabsichtigt kommunikative Spielräume eingeschränkt haben.

% Die Forschungsfragen lassen sich insgesamt wie folgt beantworten:

% \begin{enumerate}
%     \item \textbf{Forschungsfrage}: Eine spielbasierte Umgebung zur Verbesserung der Kommunikation ist grundsätzlich realisierbar, zeigt jedoch erst bei größeren Stichproben klare Wirkungspotenziale.
%     \item \textbf{Forschungsfrage}: Eine solche Umgebung muss insbesondere asymmetrische Informationsverteilung, kollaborative Problemstellungen und immersive Rollen ermöglichen um Gesprächsführungen, Gesprächsinitiativen, Empathie, soziale Präsenz und Verständnissicherungen zu adressieren.
%     \item \textbf{Forschungsfrage}: Verbesserungen in der Kommunikation konnten in Ansätzen beobachtet werden (z.B. veränderte Gesprächsführung und Pausenzeiten), blieben jedoch unterhalb des signifikanten Niveaus.
%     \item \textbf{Forschungsfrage}: Unterschiede im Kommunikationsverhalten konnten leider nicht beobachtet werden, da eine \ac{AR}-Anwendung nicht funktionsfähig umgesetzt werden konnte und es diesbezüglich auch keine Versuchsdurchführungen gab.
%     \item \textbf{Forschungsfrage}: Die befragten Probanden standen dem spielerischen Ansatz offen gegenüber; insbesondere die Hemmschwellen im Umgang mit freunden Personen wurde tendenziell gesenkt.
% \end{enumerate}

Die Forschungsfragen lassen sich insgesamt wie folgt zusammenfassen:
Eine spielbasierte Umgebung zur Verbesserung der Kommunikation ist grundsätzlich realisierbar, entfaltet ihr volles Wirkungspotenzial jedoch erst bei größeren Stichproben und vermutlich auch längerem Spielinhalt. Damit eine solche Umgebung kommunikativ wirksam ist, muss sie insbesondere eine asymmetrische Informationsverteilung, kollaborative Problemstellungen und immersive Rollen bieten. Nur so können zentrale Kommunikationsparameter wie Gesprächsführung, Gesprächsinitiativen, Empathie, soziale Präsenz und Verständnissicherungen gezielt angesprochen werden. In der durchgeführten Studie konnten erste Hinweise auf Verbesserungen in der Kommunikation beobachtet werden, etwa in Form veränderter Gesprächsführung und kürzerer Pausenzeiten, wenngleich diese unterhalb des statistisch signifikanten Niveaus bleiben. Unterschiede im Kommunikationsverhalten bei Verwendung einer \ac{AR}- und \ac{3D}-Anwendung konnten hingegen nicht analysiert werden, da die geplante AR-Anwendung nicht funktionsfähig umgesetzt wurde und entsprechende Versuchsdurchführungen ausbleiben. Dennoch zeigten sich die befragten Probanden dem spielerischen Ansatz gegenüber offen; insbesondere ließ sich eine tendenzielle Reduktion von Hemmschwellen im Umgang mit nicht vertrauten Personen feststellen.