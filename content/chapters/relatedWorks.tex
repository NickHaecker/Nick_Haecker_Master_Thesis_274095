\chapter{Stand der Forschung}

% neuer plan: zunächst die beiden referenz paper zusammenfassen, dann die methodik für die litertaur recherche nennen; dann die ergebnisse als related works für die herleitung auch der forschungsfragern beantworten und dann noch die anderen paper die so gefunden wurden noch nennen; das für den ersten teil bei der forschungsfragen beanwtortet werden ,was alles relevant für die weitere entwicklung ist

% \section{Verwandte Arbeiten}\label{sec:related-works}
In der \ac{CSCW} und \ac{HCI} bzw. \ac{CHI} existieren bereits diverse Arbeiten zu Multiplayer-Spielen, Effekt von Computerspiele auf das soziale Beisammensein der Spieler und welche Aspekte aus dem Gamedesign dafür verantwortlich sind.

Im Zentrum steht dabei häufig die Frage, wie Spiele soziale Interaktionen fördern oder hemmen, sowohl im kompeteiven als auch im kollaborativen Kontexten.

Video- und Computerspiele im Allgemeinen können einen positiven Einfluss auf das Miteinander haben. So untersuchte \cite{mason_friends_2013} wie wichtig Freundschaften für den Erfolg von Einzelpersonen und Teams in komplexen kollaborativen Umgebungen sind. Sie fanden heraus, dass Freundschaften einen großen Einfluss auf die verbesserte individuelle- und Teamleistung haben. Spieler richten sich dabei nach sozialen Gelegenheiten aus, sodass verborgene Freundschaftsbeziehungen direkt abgeleitet werden konnten. Kern der Studie war dabei der Online Multiplayer First-Person-Shooter \say{Halo: Reach} bei dem Spieler des Spiels eine anonyme Online Umfrage Fragen ausfüllen mussten. 

Doch soziale Dynamiken verlaufen nicht immer so positiv wie erhofft. Andere Untersuchungen zeigen ein differenzierteres Bild des Zusammenspiels in Onlinewelten.

So argumentiert \cite{ducheneaut_alone_2006} anhand einer Langzeitstudie zu \say{World of Warcraft}, dass soziale Aktivitäten in \ac{MMOG}s, oft überschätzt werden. Die meisten Spieler sind zwar von anderen umgeben, interagieren jedoch nur selten aktiv miteinander. Sie spielen häufig \say{allein zusammen}. Vor allem in den Quests zum Anfang ist das oft der Fall. Erst durch langfristige soziale Strukturen wie Gilden entstehen nachhaltige Bindungen und echte Zusammenarbeit.

Damit jedoch solche sozialen Beziehungen überhaupt entstehen können, ist es essenziell, dass Spiele die Aufmerksamkeit und das Interesse der Spielenden wecken - ein Aspekt, der unter dem Begriff \say{Player Engagement} intensiv erforscht wird.

\cite{rashed_review_2025} fassen in ihrer Überblickasarbeit unterschiedliche Methoden zur Schätzung des Spieler-Engagements zusammen. Ihr Ziel war es, über verschiedene Messmethoden wie EEG, Mimik, Eye Tracking und Spieler-Verhalten hinreichend eindeutige Daten zu sammeln um darüber eine Aussage über das Engagement treffen zu können. Die Validierung der Ergebnisse, da das Engagement subjektiv ist, ist schwer um objektiv eine \say{Ground Truth} Aussage treffen zu können.  \cite{yu_video_2023} verfolgten einen anderen Ansatz. Sie versuchten nicht nur auf das Engagement der Spieler einzugehen, sondern erforschten direkt im Bereich Zusammenarbeit und Kollaborative Fähigkeiten. Sie untersuchten kommerzielle Multiplayer-Spiele um Konzepte und Spielmechaniken zu identifizieren, die von Game-Designern zur Förderung von kooperativen Spielen genutzt werden können. Im Zuge der Forschung entwickelten sie kleine Prototypen und führten mit ihnen kleine Studien durch. 

Einige Studien gehen noch einen Schritt weiter und untersuchten nicht nur Engagement, sondern die spezifischen Bedingungen erfolgreicher Kooperation in Spielen - insbesondere durch das Design asymmetrischer Rollenverteilungen.

So zeigen die Arbeiten von \cite{harris_beam_2014}, \cite{harris_leveraging_2016} und \cite{harris_asymmetry_2019}, dass asymmetrische Spielkonzepte - bei denen sich Rollen, Fähigkeiten und Ziele der Spielenden unterschieden - einen positiven Einfluss auf die Zusammenarbeit hat. Untersucht werden dabei die Faktoren \say{Interdependence}, \say{Degrees of Interdepencene} sowie Mechaniken der Asymmetrie und Abhängigkeiten der Anwendungen. Ein asymmetrisches Spielkonzept ermöglicht außerdem eine Integration bzw. Inklusion von Spielergruppen mit eingeschränkten Fähigkeiten (vgl. \cite{goncalves_exploring_2021}). Für die Entwicklung von Spielen, die für die gesamte Familie gedacht sind, eignet es sich ebenfalls (vgl. \cite{pais_promoting_2024}).

Die Arbeiten von Harris et. al. dienen als Grundlage für die weitere Entwicklung des Game Designs für asymmetrische Multiplayer-Spiele. So identifizierte \cite{guimaraes_rocha_game_2008} verschiedene kooperative Design-Pattern, die in der weiteren Forschung und deren Spielumsetzung anklang fanden. In der Arbeit von \cite{emmerich_impact_2017} werden drei der definierten Pattern verwendet um eine Aussage darüber treffen zu können, wie sich Interaktionen im Spiel gezielt gestalten lassen. Die Ergebnisse der Studie zeigen, dass eine hohe Spielerinterdependenz mit mehr Kommunikation und weniger Frustration einhergeht. Geteilte Kontrolle führte jedoch zu einem geringeren Erleben von Kompetenz und Autonomie.

Diese gestalterischen Grundlagen bilden einen Ausgangspunkt für eine weiterführende Forschung, die sich nun den sozialen, psychologischen und metastrukturellen Wirkungen dieser Spielkonzepte widmet.

Die Arbeit von \cite{depping_trust_2016} beschäftigte sich mit dem zwischenmenschlichen Vertrauen innerhalb einer zusammenarbeitenden Gruppe. Der Fokus lag dabei auf der Problematik, dass im Online-Umfeld bewährte Methodiken zum Teambildung nur schwer umsetzbar sind und bestimmte Situationen einfacher simuliert werden müssen. Daher wurde durch Einsatz eines sozialen Spiels bestimmte Situation wie Risikosituationen und gegenseitige Abhängigkeiten simuliert. Das Zusammenarbeiten im Team kann auch eine Quelle von Konflikten oder Veränderungen sein. \cite{velez_ingroup_2014} zeigen den Fall, dass eine (neue) fremde Person zu einer bestehenden Gruppe Spannungen erzeugen kann. Ihre Studie belegtr, dass kooperative Spiele nicht nur das Helferverhalten steigert, sondern auch das Aggressionsverhalten gegenüber Mitgliedern einer Fremdgruppe verringern kann.

In der Forschung von \ac{VR}-Spielen entstanden einige Interessante Arbeiten bezüglich des Game-Designs aber auch der enthaltenen Forschung.

\cite{karaosmanoglu_playing_2023} untersuchten die Vertrautheit von Zweierteams, die aus sich Fremden oder befreundeten Personen bestanden, im Zusammenhang mit sozialen und spielerischen Erfahrungen sowie ihrer Spielleistung. Die Studie ergab, dass es keine signifikanten Unterschiede zwischen den Freundeteams und Fremdenteams gab. Um Zusammenarbeit ging es ebenfalls in der Anwendung von \cite{sajjadi_maze_2014}. Die Ergebnisse der Studie zeigen, dass das konzipierte Spielkonzept bei den Spieler-Rollen mit den Sifteo Cubes und der VR Anwendung für die Oculus Rift eine positive Bewertung sowohl des Spielerlebnisses als auch der Zusammenarbeit ergab. Ebenfalls mit dem Bezug auf die Zusammenarbeit beschäftigte sich die Arbeit von \cite{smilovitch_birdquestvr_2019}, bei der es darüber hinaus um das Ausschöpfen der Möglichkeiten von \ac{VR} ging.

Im Kerngebiet der Kommunikation beschäftigte sich \cite{nasir_cooperative_2013} und \cite{nasir_effect_2015} zunächst mit der Entwicklung eines \say{ice-breaking} Spiels, das in Form eines 2D-\ac{RPG} konzipiert und entwickelt wurde. Der Sinn des Spiels ist dabei, die Zusammenarbeit in einer folgenden Gruppenarbeit zu verbessern. In der Studie wurden dabei drei unterschiedliche Gruppen miteinander verglichen (eine Gruppe hat das konzipierte Spiel gespielt, eine weitere hatte ein generisches ice-breaking Spiel gespielt und die dritte Gruppe keins). Die Gruppen, die das konzipierte Spiel gespielt hatten, zeigten eine erhöhte Interaktion. Die fortführende Studie untersuchte, ob das aus der ersten Studie umgesetztee Spiel die Zusammenarbeit in realen Teams verbessern kann. Es wurden dabei Gruppen verglichen, die vor der Arbeitsaufgabe das konzipierte ice-breaking gespielt hatten, mit denen, die es nicht gespielt hatten. Es wurde festgestellt, dass die Gruppen, die das ice-breaking Spiel spielten, in der anschließenden Arbeitsaufgabe eine erhöhte Zusammenarbeit zeigten.
% zunächst mit der Wirkung eines kooperativen \say{ice-breaking} Spiel (Kennenlern-Spiel), das als Instrument dienen soll die Zusammenarbeit zu verbessern. Die Ergebnisse der Studie zeigten, dass die Gruppe, die das Kennenlern-Spiel gespielt hatten, mit erhöhter Interaktion an der Folgeaufgabe teilgenommen hatten, als die Vergleichsgruppe. In der darauffolgenden Arbeit 

% Die \ac{AR}-Forschung beschäftigte sich ebenfalls mit 

% [weiterführen dann mit den papern von den design pattern, da die hier dazu kommen, dann beispiele bringen die das als grundlage nehmen im übertragenene sinne]

% Eine der Arbeiten im Themengebiet dieser Arbeit wurde von \cite{nasir_effect_2015}, bei der es darum geht eine \say{Icebreaking}-Anwendung in Form eines Computer- und Videospiels zu haben um erste Hürden im Kennenlernen und effektiven gemeinsamen Arbeiten zu überwinden. Es zeigte sich, dass die Gruppe, die das Icebreaking-Videospiel gespielt hatte, eine erhöhte Zusammenarbeit. [Fortsezen mit inhalt des spiels und dann die forschung beschreiben]

% \cite{harris_asymmetry_2019}
% \cite{sajjadi_maze_2014}

% hier würden Paper reinkommen die asymmetrische Multiplayer gemacht haben, welche aspekte da mitreinspielen, da kommen dann auch die wichtigen Begriffe dazu mitrein. Auch bereits umgesetzt asymetrische VR Spiele?


% Auch Anna Lotz´ Thesis wäre hier relevant


\section{Wichtige Begriffe}
In den vorangegangenen Arbeiten beschäftigten sich die Autoren mit einigen Begrifflichkeiten, die Grundlage in der Konzeption und Entwicklung dieses Prototyps sowie der Forschung dieser Arbeit sind. 

In den folgenden Kapiteln werden diese Begriffe erklärt.

\subsection{Interdependence}
Der Begriff \say{Interdependence} stammt aus dem psychologischen Rahmenwerk für soziale und gruppenbezogene Interaktionen. Die Interdependence wird über das Ausmaß, in dem Gruppenmitglieder aufeinander angewiesen sind, um ihre Aufgabe effektiv zu erfüllen, definiert \cite[S. 451]{depping_cooperation_2017}, \cite{saavedra_complex_1993}, \cite[S. 197:4]{holly_asymmetric_2023}. Auf Video- und Computerspiele bezogen, können Aufgaben als das Spielziel bezeichnet werden \cite[S. 451]{depping_cooperation_2017}. 
In \cite[S. 52]{van_der_vegt_patterns_2001} werden unterschiedliche Formen der Interdependence vorgestellt:
\paragraph{Task interdependence} beschreibt die Abhängigkeit von Teammitgliedern in ihren Aufgaben, die sie zu tun haben. Der Grad der Abhängigkeit nimmt zu, je komplexer die Aufgabe wird.
\paragraph{Goal interdependence} beschreibt die quantitativen und qualitativen Leistungen, die von den Gruppenmitgliedern gemeinsam erreicht werden müssen, um das Gruppenziel zu erreichen.

[Hier schauen ob noch in den anderen Papern was dazu steht]

% \begin{itemize}
% \item \textbf{Task interdependence}: 
%     % \item \textbf{Kann eine spielbasierte Umgebung für die Untersuchung und Verbesserung von Kommunikation zwischen zwei oder mehreren Personen realisiert werden?}
%     % \item \textbf{Welche spezifischen Eigenschaften muss eine solche Umgebung aufweisen und welche Kommunikationsparameter werden dabei angesprochen?}
%     % \item \textbf{Welche Verbesserungen in der Kommunikation zwischen den Anwendern können durch ein asynchrones Multiplayer-Spiel mit zwei verschiedenen Spielerklassen beobachtet werden?}
%     % \item \textbf{Welche Unterschiede können in der Art das Kommunikationsverhalten bei der Verwendung von zwei unterschiedlichen Anwendungen (AR und 3D) (festgestellt/beobachtet) werden}
%     % \item \textbf{Wie stehen die Nutzer zu einem spielerischen Ansatz und zur Verbesserung der Kommunikation, insbesondere auch im Umgang mit Fremden?}
% \end{itemize}
% \cite{harris_leveraging_2016}
% \cite{depping_cooperation_2017}

\subsection{Degrees of Interdependence}
In der Arbeit von \cite[S. 7]{harris_asymmetry_2019} werden unterschiedliche Grade der Interdependence untersucht. Unter Grad der interdependence versteht man das Ausmaß in dem die Handlungen der Spieler voneinander abhängig sind, um das Spielziel erfolgreich zu erreichen. Je höher der Grad der Interdependence, desto stärker sind die Spieler darauf angewiesen, sich gegenseitig abzustimmen zusammenzuarbeiten um ihre Handlungen aufeinander abzustimmen (vgl. \cite[S. 7]{harris_asymmetry_2019}.

[Hier schauen ob noch in den anderen Papern was dazu steht]

% \cite{beznosyk_effect_2012}

\subsection{Soziale Präsenz}
Die soziale Präsenz beschreibt \say{das Gefühl, mit einem anderen zusammen zu sein} [eigene Übersetzung] \cite[S. 1]{biocca_towards_2003}. \say{Das andere} kann dabei entweder ein anderer Mensch oder eine künstliche Intelligenz sein. Innerhalb der \ac{HCI} untersucht die Theorie der sozialen Präsenz, wie das \say{Gefühl, mit einem anderen anderen zusammen zu sein} [eigene Übersetzung] \cite[S. 1]{biocca_towards_2003} durch Schnittstellen gestaltet und beeinflusst wird (vgl. \cite[S. 1]{biocca_towards_2003}). Sie wird im Einzelnen durch die Wahrnehmung der physischen Repräsentation des anderen Spielers sowie durch psychologische Beteiligung und Verhaltensabhängigkeiten gekennzeichnet. Soziale Präsenz kann somit als das Ergebnis eines komplexen Zusammenspiels von wechselseitigem Verhalten, Kommunikation und sozialen Kontextmerkmalen gesehen werden. Die Voraussetzung hierfür ist, dass ein Spieler die Kopräsenz einer anderen sozialen Einheit wahrnimmt (vgl. \cite[S. 1]{emmerich_game_2016}).

[Hier schauen ob noch in den anderen Papern was dazu steht]

% \cite{emmerich_impact_2017}

% Vertrauen gibts in dem Kontext auch und wie man dan über Spiele aufbaut

% \section{}#

\section{Forschungsbeitrag}

% In diesem Kapitel wird der Untersuchungsschwerpunkt dieser Arbeit vorgestellt. Dabei wird Bezug auf die bestehende Forschung aus dem Kapitel \emph{\nameref{sec:related-works}} genommen.

Diese Arbeit greift die grundlegende Forschung der Arbeiten von Nassir in \cite{nasir_cooperative_2013} und \cite{nasir_effect_2015} auf und erweitert diese durch Vor- und Nachtests derselben Versuchsgruppe, um zu beweisen, dass eine bestehende Gruppe durch die Anwendung des erstellten Prototyps gezielt in der gemeinsamen Kommunikation verbessert werden kann. Der Prototyp dient zudem nicht als stilisierte Anwendung für den Zweck des Ice-Breakings, sondern auch als Multiplayer-Spiel, das in der Freizeit gespielt werden kann.

Hinzu kommt, dass die Anwendung ein \say{Cross-Plattform} Multiplayer (vgl. Abbildung: \ref{fig:lotz_multiplayer_types}) ist, bei dem unterschiedliche Plattformen genutzt werden müssen und die damit einhergehende Wirkung mit untersucht werden soll. Im Fokus steht dabei die \ac{AR}-Integration einer Anwendung und die Touchsteuerung beider Anwendungen, die im Prototyp umgesetzt werden sollen.


\chapter{Stand der Technik} \label{sec:sota}

% \section{Artverwandte Spiele}
Nachdem die einzelnen Charakteristiken von Multiplayer-Spielen im Kapitel \emph{\nameref{sec:basics}} wurden, werden nun Spiele vorgestellt, welche im Rahmen dieser Arbeit näher betrachtet wurden.

Die Spielreihe \say{\textbf{We were here}} vom niederländischen Entwicklerstudio Total Mayhem Games beinhaltet asymmetrische Kooperative-Multiplayer-Spiele, bei denen zu zweit Rätsel und Hindernisse in der Spielwelt gelöst werden müssen um aus der Umgebung, in denen die Avatare der Spieler gefangen sind, zu entkommen. Dabei können die Spieler über ein \say{In-Game}-Walki-Talki miteinander kommunizieren. Zumeist ist es so, dass ein Spieler verschiedene Rätsel oder Hindernisse für sich hat, die er seinem Mitspieler beschreiben muss, damit dieser die passenden Antworten übermitteln oder Rätsel lösen kann. Die beiden Spieler befinden sich dabei in abgetrennten Räumen oder Gebieten innerhalb der Spielwelt (vgl. \cite{noauthor_we_nodate}; \cite{noauthor_total_nodate}).  

[hier it takes two und split fiction erwähnen]

Das Spiel \say{\textbf{The past within}} vom ebenfalls aus den Niederlanden kommenden Entwicklerstudio Rusty Lake ist ein asymmetrisches kooperatives Multiplayer-Spiel bei dem zwei Spieler gemeinsam in einer Sitzung sowohl in der Vergangenheit als auch in der Zukunft gemeinsam Rätsel lösen müssen um der Protagonistin und ihrem Vater zu helfen. Jeweils ein Spieler befindet sich dabei in einer 2D-Amnwendung, der andere in einer 3D-Anwendung. Es existiert die Möglichkeit, dass das Spiel von verschiedenen Plattformen aus gespielt werden kann (Cross-Plattform Spielbarkeit) (vgl. \cite{noauthor_past_nodate}). 

Das bereits in den vorangegangenen Kapiteln [Kapitel einbinden] erwähnte \say{\textbf{Keep Talking and Nobody Explodes}} ist ein asymmetrisches kooperatives Multiplayer-Spiel bei dem eine Person das Spiel besitzen muss damit es im Team gespielt werden kann. Das Spiel hat eine Besonderheit, da es ein Cross-Plattform Spiel ist, bei dem ein Teilnehmer (der Bombenentschärfen) eine Bombe entschärfen muss und die anderen Spielteilnehmer (die Experten) verschiedene Anleitungen von Bomben vorliegen haben. Die Aufgabe besteht darin die richtige Anleitung für die entsprechende Bombe zu finden und die Bombe innerhalb der vorgegebenen Zeit zu entschärfen. Im Spiel befindet sich jedoch nur der Bombenentschärfer, während die Experten die Anleitungen ausgedruckt durchschauen können (vgl. \cite{noauthor_keep_nodate}).

Im März 2025 erschien das Spiel \say{\textbf{Myrmidon}} vom Studio Studio Popot, welches ein asymmetrischer kooperativer Multiplayer ist, bei dem zwei Spieler zusammen, in zwei verschiedenen Rollen, miteinander spielen können. Eine Rolle ist dabei die Stop-Motion Puppe, welche in einer Stop-Motion Welt Hindernisse überqueren und über verschiedene Plattformen springen muss um ans Ziel zu kommen. Unterstützt wird die Puppe dabei vom Animator, der die Kulisse des Stop-Motion Films bedienen muss, damit die Puppe an ihr Ziel gelangt (vgl. \cite{noauthor_myrmidon_2024}).


% \section{Kooperative Gamedesign Pattern}
% FÜr die methodiische arbeit wichtig
% \subsection{Game Design Patterns}
% \cite{bjork_patterns_2005}

% \paragraph{Complementarity}

% \paragraph{Synergies}

% \paragraph{Abilities}

% \paragraph{Shared Goals}

% \paragraph{Synergies between goals}

% \paragraph{Special Rules for Player of the same Team}

% \paragraph{Camera Setting}

% \paragraph{Interacting with the same object}

% \paragraph{Shared puzzle}

% \paragraph{Shared characters}

% \paragraph{Special characters targetting lone wolf}

% \paragraph{Vocalization}

% \paragraph{Limited ressources}

% \paragraph{Einflussnahme}


% taxonomien Towards a Unified Taxonomy for Analog and Digital Escape Room Games hier vcrstellen; generell wie in der konzeption vorgegnagen wird