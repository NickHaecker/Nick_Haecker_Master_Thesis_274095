\chapter{Fazit}
% Ganzheitlich konnte mit der konzeptionellen Weiterentwicklung und Überarbeitung der bestehenden Anwendung ein 
Ziel dieser Masterarbeit war es, mit Connecting-Minds einen Spiel-basierten Prototypen zu entwickeln, der als Versuchsumgebung zur Untersuchung kommunikativer Prozesse in asymmetrischen Multiplayer-Szenarien dient. Die zentrale Forschungsfrage lautet, inwiefern durch ein solches Spielkonzept die Kommunikation zwischen zwei Personen, insbesondere im Umgang mit zunächst fremden Personen, verbessert werden kann.

Die Analyse der quantitativen und qualitativen Daten zeigt, dass der Prototyp Potenzial besitzt, um soziale Nähe zwischen Spielpartnern zu fördern. Zwar konnten aufgrund der geringen Stichprobengröße keine signifikanten Veränderungen im Kommunikationsverhalten nachgewiesen werden, jedoch deuten einzelne Effekte - wie etwa die mittleren Korrelationen bei Gesprächsinitiativen oder der Pausenzeiten - auf erste positive Tendenzen hin. Darüber hinaus wurde das Spielkonzept von den Probanden überwiegend positiv bewertet. insbesondere in Bezug auf auf seine Originalität und motivierende Wirkung.

Gleichzeitig wurden auch Schwächen offensichtlich: Vor allem die Bedienbarkeit der Watcher-Anwendung wurde vielfach kritisch bewertet. Eine klarere Trennung der Gestensteuerung (Zoom vs. Rotation) sowie eine konsistente UI-Gestaltung stellen zentrale Ansatzpunkte für zukünftige Iterationen dar. Auch konzeptionell wurden einige Rätsel als nicht vollständig durchdacht empfunden und sollten überarbeitet werden, um das das interdependente Zusammenspiel der beiden Rollen stärker herauszuarbeiten.

Die Forschungsfragen lassen sich insgesamt wie folgt beantworten:

\begin{enumerate}
    \item \textbf{Forschungsfrage}: Eine spielbasierte Umgebung zur Verbesserung der Kommunikation ist grundsätzlich realisierbar, zeigt jedoch erst bei größeren Stichproben klare Wirkungspotenziale.
    \item \textbf{Forschungsfrage}: Eine solche Umgebung muss insbesondere asymmetrische Informationsverteilung, kollaborative Problemstellungen und immersive Rollen ermöglichen um Gesprächsführungen, Gesprächsinitiativen, Empathie, soziale Präsenz und Verständnissicherungen zu adressieren.
    \item \textbf{Forschungsfrage}: Verbesserungen in der Kommunikation konnten in Ansätzen beobachtet werden (z.B. veränderte Gesprächsführung und Pausenzeiten), blieben jedoch unterhalb des signifikanten Niveaus.
    \item \textbf{Forschungsfrage}: Unterschiede im Kommunikationsverhalten konnten leider nicht beobachtet werden, da eine \ac{AR}-Anwendung nicht funktionsfähig umgesetzt werden konnte und es diesbezüglich auch keine Versuchsdurchführungen gab.
    \item \textbf{Forschungsfrage}: Die befragten Probanden standen dem spielerischen Ansatz offen gegenüber; insbesondere die Hemmschwellen im Umgang mit freunden Personen wurde tendenziell gesenkt.
\end{enumerate}

\section{Ausblick}\label{sec:prospect}
% Die Arbeit legt die Grundlage für weiterführende Forschung im Bereich kollaborativer Kommunikation durch asymmetrische Multiplayer Spiele. Zukünftige Studien sollten insbesondere die Stichprobenzahl erhöhen, um signifikante Aussagen treffen zu können, mehrere Zwischenmessungen einbauen, um differenziertere Entwicklungsverläufe erfassen zu können

Diese Arbeit legt die Grundlage für weitere Forschungen im Bereich kollaborativer Kommunikation durch asymmetrische Multiplayer Adventure-Spiele.
% Zukünftige Studien sollten insbesondere: 
Aufbauend auf den gewonnenen Erkenntnissen sollten zukünftige Studien eine größere Stichprobenzahl einbeziehen, um statistisch signifikante Aussagen treffen zu können. Darüber hinaus wäre es sinnvoll, Zwischenmessungen zu integrieren, um Entwicklungsverläufe differenzierter erfassen und interpretieren zu können. Ergänzend könnten noch weitere Erhebungsinstrumente, etwa Fragebögen zur wahrgenommenen Interdependenz oder der qualitativen Usability, eingesetzt werden, um den Prototyp gezielter weiterentwickeln zu können. Auch das bestehende Rätsel- und Interface-Design bietet Potenzial für iterative Weiterentwicklungen, insbesondere im Hinblick auf Benutzerfreundlichkeit, Rollenbalance und die spielmechanische Unterstützung kollaborativer Zusammenarbeit.

% Das Konzept bietet einen vielversprechenden 
Das Konzept Connecting-Minds bietet nicht nur eine spannende Möglichkeit, um das Kommunikationsverhalten von Dyaden zu verbessern, sondern auch einen allgemeinen Spielspaß mit einer neuartigen Spielmechanik. Insofern die Gebrauchstauglichen Aspekte verbesser wurden und die Konzeption final umgesetzt wurde, steht dem Spiel nichts im Weg es der breiten Öffentlichkeit zugänglich zu machen.