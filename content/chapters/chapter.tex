\chapter{Konzeption}
In diesem Kapitel wird die Konzeption des Prototyps vorgestellt.
% Sie enthält zunächst das methodische Vorgehen des Abschnitts, im Anschluss folgt eine Analyse der Spiele, die im Fokus der Arbeit stehen und aus welchen Aspekte in die Konzeption eingeflossen sind. Der Hauptteil der Konzeption umfassen das Game-Design des Prototyps.
% \cite{harris_leveraging_2016}
% .\cite{van_der_vegt_patterns_2001}

% \section{Methodisches Vorgehen}


\section{Analyse verwandter Spiele}

\section{Genre}

\section{Spielmechanik}

\section{Spielablauf}

\subsection{Spielablauf des Spiels}

\subsection{Levelablauf}

\section{Session}

% \section{Belohnungen}

\section{Spielerrollen}

\subsection{Player}

\subsubsection{Interaktion mit Gegenständen}

\subsubsection{Tragen von Gegenständen}

\subsection{Watcher}

\subsubsection{Platzieren von Gegenständen}

\subsubsection{Entfernen von Gegenständen}

\subsubsection{Previewen von Gegenständen}

\section{Gegenstände}

\subsection{Leichte Gegenstände}

\subsection{Schwere Gegenstände}

\subsection{Hinweise}

\section{Leveldesign}

\subsection{Gegenstände}

\subsection{Hinweise}

\subsection{Hindernisse}

\section{Informationen für den Spieler}

\section{Sounddesign}

\chapter{Visuelles Design des Prototyps}

\section{Moodboard}

\section{Art-Stil}

\section{Avatar des Players}

\section{User Interface}

\section{Führung durch das Level}

\section{Gegenstände}

\section{Menü}

\section{Leveldesign}

\chapter{Umsetzung des Prototyps}

\section{Verwendete Technologien}

\section{Ausgangssituation}

\section{Aufbau des Prototyps}

\section{Herausforderungen in der Umsetzung}